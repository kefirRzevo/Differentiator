\documentclass[12pt]{article}                             
\usepackage[utf8]{inputenc}                               
\usepackage[russian]{babel}                               
\usepackage{geometry}                                     
\usepackage{amsmath}                                      
\geometry{a4paper}                                        
\usepackage{graphicx}                                     
\begin{document}                                          
\begin{titlepage}                                         
	\begin{center}                                          
		\Huge                                                 
		Сан-Юрьевич                                            
		\vspace*{1cm}                                         
                                                           
		\textbf{Дифференциальный зачёт}                       
                                                           
		\vspace{0.5cm}                                        
		\vspace{1.5cm}                                        
		\includegraphics[width = 10 cm]{imgs/petrovich.png}   
		\begin{minipage}{10cm}                                
		\vspace*{2cm}                                         
			\begin{center}                                      
				Из МФТИ, с любовью                                 
			\end{center}                                        
		\includegraphics[width = 3 cm]{imgs/with_love.png}    
		\end{minipage}                                        
                                                           
	\end{center}                                            
\end{titlepage}                                           
\newpage                                                  
\Huge                                                     
	Посчитаем производную\\                                
\newline                                                  
\normalfont                                               
\normalsize                                               
А Флуктуационно-диссипационная теорема гласит, что:  \begin{equation}
	\frac{\partial}{\partial x}\left( \frac{\partial}{\partial x}\left( \frac{\sin {x^{5}}}{\cos {\left( 5\cdot x\right) ^{3}}}\right) \right) 
\end{equation}
Очевидно, что по критерию Сильвестра:  \begin{equation}
	\frac{\partial}{\partial x}\left( \frac{\frac{\partial}{\partial x}\left( \sin {x^{5}}\right) \cdot \cos {\left( 5\cdot x\right) ^{3}} - \sin {x^{5}}\cdot \frac{\partial}{\partial x}\left( \cos {\left( 5\cdot x\right) ^{3}}\right) }{\left( \cos {\left( 5\cdot x\right) ^{3}}\right) ^{2}}\right) 
\end{equation}
Слава Украине, героям слава:  \begin{equation}
	\frac{\partial}{\partial x}\left( \frac{A - B}{\left( \cos {\left( 5\cdot x\right) ^{3}}\right) ^{2}}\right) 
\end{equation}
Где, 

\begin{equation}
	B = 
\sin {x^{5}}\cdot \frac{\partial}{\partial x}\left( \cos {\left( 5\cdot x\right) ^{3}}\right) 
\end{equation}
\begin{equation}
	A = 
\frac{\partial}{\partial x}\left( x^{5}\right) \cdot \cos {x^{5}}\cdot \cos {\left( 5\cdot x\right) ^{3}}
\end{equation}
А по теореме Лиувилля об интеграле уравнения Гамильтона — Якоби:  \begin{equation}
	\frac{\partial}{\partial x}\left( \frac{A\cdot \cos {\left( 5\cdot x\right) ^{3}} - B}{\left( \cos {\left( 5\cdot x\right) ^{3}}\right) ^{2}}\right) 
\end{equation}
Где, 

\begin{equation}
	B = 
\sin {x^{5}}\cdot \frac{\partial}{\partial x}\left( \cos {\left( 5\cdot x\right) ^{3}}\right) 
\end{equation}
\begin{equation}
	A = 
\frac{\partial}{\partial x}\left( x\right) \cdot 5\cdot x^{5 - 1}\cdot \cos {x^{5}}
\end{equation}
Как известно, по теореме Пифагора:  \begin{equation}
	\frac{\partial}{\partial x}\left( \frac{A\cdot \cos {\left( 5\cdot x\right) ^{3}} - B}{\left( \cos {\left( 5\cdot x\right) ^{3}}\right) ^{2}}\right) 
\end{equation}
Где, 

\begin{equation}
	B = 
\sin {x^{5}}\cdot \frac{\partial}{\partial x}\left( \cos {\left( 5\cdot x\right) ^{3}}\right) 
\end{equation}
\begin{equation}
	A = 
1\cdot 5\cdot x^{5 - 1}\cdot \cos {x^{5}}
\end{equation}
А Флуктуационно-диссипационная теорема гласит, что:  \begin{equation}
	\frac{\partial}{\partial x}\left( \frac{A\cdot \cos {\left( 5\cdot x\right) ^{3}} - \sin {x^{5}}\cdot B}{\left( \cos {\left( 5\cdot x\right) ^{3}}\right) ^{2}}\right) 
\end{equation}
Где, 

\begin{equation}
	B = 
-1\cdot \frac{\partial}{\partial x}\left( \left( 5\cdot x\right) ^{3}\right) \cdot \sin {\left( 5\cdot x\right) ^{3}}
\end{equation}
\begin{equation}
	A = 
1\cdot 5\cdot x^{5 - 1}\cdot \cos {x^{5}}
\end{equation}
Гаусс еще в \RomanNumeralCaps{14} веке посчитал, что:  \begin{equation}
	\frac{\partial}{\partial x}\left( \frac{A\cdot \cos {\left( 5\cdot x\right) ^{3}} - \sin {x^{5}}\cdot -1\cdot B\cdot \sin {\left( 5\cdot x\right) ^{3}}}{\left( \cos {\left( 5\cdot x\right) ^{3}}\right) ^{2}}\right) 
\end{equation}
Где, 

\begin{equation}
	B = 
\frac{\partial}{\partial x}\left( 5\cdot x\right) \cdot 3\cdot \left( 5\cdot x\right) ^{3 - 1}
\end{equation}
\begin{equation}
	A = 
1\cdot 5\cdot x^{5 - 1}\cdot \cos {x^{5}}
\end{equation}
Как известно, по теореме Пифагора:  \begin{equation}
	\frac{\partial}{\partial x}\left( \frac{A\cdot \cos {\left( 5\cdot x\right) ^{3}} - \sin {x^{5}}\cdot -1\cdot \left( B\right) \cdot 3\cdot \left( 5\cdot x\right) ^{3 - 1}\cdot \sin {\left( 5\cdot x\right) ^{3}}}{\left( \cos {\left( 5\cdot x\right) ^{3}}\right) ^{2}}\right) 
\end{equation}
Где, 

\begin{equation}
	B = 
\frac{\partial}{\partial x}\left( 5\right) \cdot x + 5\cdot \frac{\partial}{\partial x}\left( x\right) 
\end{equation}
\begin{equation}
	A = 
1\cdot 5\cdot x^{5 - 1}\cdot \cos {x^{5}}
\end{equation}
Слава Украине, героям слава:  \begin{equation}
	\frac{\partial}{\partial x}\left( \frac{A\cdot \cos {\left( 5\cdot x\right) ^{3}} - \sin {x^{5}}\cdot -1\cdot B\cdot \sin {\left( 5\cdot x\right) ^{3}}}{\left( \cos {\left( 5\cdot x\right) ^{3}}\right) ^{2}}\right) 
\end{equation}
Где, 

\begin{equation}
	B = 
\left( 0\cdot x + 5\cdot \frac{\partial}{\partial x}\left( x\right) \right) \cdot 3\cdot \left( 5\cdot x\right) ^{3 - 1}
\end{equation}
\begin{equation}
	A = 
1\cdot 5\cdot x^{5 - 1}\cdot \cos {x^{5}}
\end{equation}
По теореме Лиувилля о сохранении фазового объёма:  \begin{equation}
	\frac{\partial}{\partial x}\left( \frac{A\cdot \cos {\left( 5\cdot x\right) ^{3}} - \sin {x^{5}}\cdot -1\cdot B\cdot \sin {\left( 5\cdot x\right) ^{3}}}{\left( \cos {\left( 5\cdot x\right) ^{3}}\right) ^{2}}\right) 
\end{equation}
Где, 

\begin{equation}
	B = 
\left( 0\cdot x + 5\cdot 1\right) \cdot 3\cdot \left( 5\cdot x\right) ^{3 - 1}
\end{equation}
\begin{equation}
	A = 
1\cdot 5\cdot x^{5 - 1}\cdot \cos {x^{5}}
\end{equation}
А по теореме Лиувилля об интеграле уравнения Гамильтона — Якоби:  \begin{equation}
	\frac{\frac{\partial}{\partial x}\left( A\cdot \cos {\left( 5\cdot x\right) ^{3}} - \sin {x^{5}}\cdot -1\cdot B\cdot \sin {\left( 5\cdot x\right) ^{3}}\right) \cdot \left( \cos {\left( 5\cdot x\right) ^{3}}\right) ^{2} - \left( C\cdot \cos {\left( 5\cdot x\right) ^{3}} - \sin {x^{5}}\cdot -1\cdot D\cdot \sin {\left( 5\cdot x\right) ^{3}}\right) \cdot \frac{\partial}{\partial x}\left( \left( \cos {\left( 5\cdot x\right) ^{3}}\right) ^{2}\right) }{\left( \left( \cos {\left( 5\cdot x\right) ^{3}}\right) ^{2}\right) ^{2}}
\end{equation}
Где, 

\begin{equation}
	D = 
\left( 0\cdot x + 5\cdot 1\right) \cdot 3\cdot \left( 5\cdot x\right) ^{3 - 1}
\end{equation}
\begin{equation}
	C = 
1\cdot 5\cdot x^{5 - 1}\cdot \cos {x^{5}}
\end{equation}
\begin{equation}
	B = 
\left( 0\cdot x + 5\cdot 1\right) \cdot 3\cdot \left( 5\cdot x\right) ^{3 - 1}
\end{equation}
\begin{equation}
	A = 
1\cdot 5\cdot x^{5 - 1}\cdot \cos {x^{5}}
\end{equation}
А по теореме Лиувилля об интеграле уравнения Гамильтона — Якоби:  \begin{equation}
	\frac{\left( \frac{\partial}{\partial x}\left( A\cdot \cos {\left( 5\cdot x\right) ^{3}}\right)  - \frac{\partial}{\partial x}\left( \sin {x^{5}}\cdot -1\cdot B\cdot \sin {\left( 5\cdot x\right) ^{3}}\right) \right) \cdot \left( \cos {\left( 5\cdot x\right) ^{3}}\right) ^{2} - \left( C\cdot \cos {\left( 5\cdot x\right) ^{3}} - \sin {x^{5}}\cdot -1\cdot D\cdot \sin {\left( 5\cdot x\right) ^{3}}\right) \cdot \frac{\partial}{\partial x}\left( \left( \cos {\left( 5\cdot x\right) ^{3}}\right) ^{2}\right) }{\left( \left( \cos {\left( 5\cdot x\right) ^{3}}\right) ^{2}\right) ^{2}}
\end{equation}
Где, 

\begin{equation}
	D = 
\left( 0\cdot x + 5\cdot 1\right) \cdot 3\cdot \left( 5\cdot x\right) ^{3 - 1}
\end{equation}
\begin{equation}
	C = 
1\cdot 5\cdot x^{5 - 1}\cdot \cos {x^{5}}
\end{equation}
\begin{equation}
	B = 
\left( 0\cdot x + 5\cdot 1\right) \cdot 3\cdot \left( 5\cdot x\right) ^{3 - 1}
\end{equation}
\begin{equation}
	A = 
1\cdot 5\cdot x^{5 - 1}\cdot \cos {x^{5}}
\end{equation}
А Флуктуационно-диссипационная теорема гласит, что:  \begin{equation}
	\frac{\left( A\cdot \cos {\left( 5\cdot x\right) ^{3}} + B\cdot \frac{\partial}{\partial x}\left( \cos {\left( 5\cdot x\right) ^{3}}\right)  - \frac{\partial}{\partial x}\left( \sin {x^{5}}\cdot -1\cdot C\cdot \sin {\left( 5\cdot x\right) ^{3}}\right) \right) \cdot \left( \cos {\left( 5\cdot x\right) ^{3}}\right) ^{2} - \left( D\cdot \cos {\left( 5\cdot x\right) ^{3}} - \sin {x^{5}}\cdot -1\cdot E\cdot \sin {\left( 5\cdot x\right) ^{3}}\right) \cdot \frac{\partial}{\partial x}\left( \left( \cos {\left( 5\cdot x\right) ^{3}}\right) ^{2}\right) }{\left( \left( \cos {\left( 5\cdot x\right) ^{3}}\right) ^{2}\right) ^{2}}
\end{equation}
Где, 

\begin{equation}
	E = 
\left( 0\cdot x + 5\cdot 1\right) \cdot 3\cdot \left( 5\cdot x\right) ^{3 - 1}
\end{equation}
\begin{equation}
	D = 
1\cdot 5\cdot x^{5 - 1}\cdot \cos {x^{5}}
\end{equation}
\begin{equation}
	C = 
\left( 0\cdot x + 5\cdot 1\right) \cdot 3\cdot \left( 5\cdot x\right) ^{3 - 1}
\end{equation}
\begin{equation}
	B = 
1\cdot 5\cdot x^{5 - 1}\cdot \cos {x^{5}}
\end{equation}
\begin{equation}
	A = 
\frac{\partial}{\partial x}\left( 1\cdot 5\cdot x^{5 - 1}\cdot \cos {x^{5}}\right) 
\end{equation}
Как известно, по теореме Пифагора:  \begin{equation}
	\frac{\left( \left( A + B\right) \cdot \cos {\left( 5\cdot x\right) ^{3}} + C\cdot \frac{\partial}{\partial x}\left( \cos {\left( 5\cdot x\right) ^{3}}\right)  - \frac{\partial}{\partial x}\left( \sin {x^{5}}\cdot -1\cdot D\cdot \sin {\left( 5\cdot x\right) ^{3}}\right) \right) \cdot \left( \cos {\left( 5\cdot x\right) ^{3}}\right) ^{2} - \left( E\cdot \cos {\left( 5\cdot x\right) ^{3}} - \sin {x^{5}}\cdot -1\cdot F\cdot \sin {\left( 5\cdot x\right) ^{3}}\right) \cdot \frac{\partial}{\partial x}\left( \left( \cos {\left( 5\cdot x\right) ^{3}}\right) ^{2}\right) }{\left( \left( \cos {\left( 5\cdot x\right) ^{3}}\right) ^{2}\right) ^{2}}
\end{equation}
Где, 

\begin{equation}
	F = 
\left( 0\cdot x + 5\cdot 1\right) \cdot 3\cdot \left( 5\cdot x\right) ^{3 - 1}
\end{equation}
\begin{equation}
	E = 
1\cdot 5\cdot x^{5 - 1}\cdot \cos {x^{5}}
\end{equation}
\begin{equation}
	D = 
\left( 0\cdot x + 5\cdot 1\right) \cdot 3\cdot \left( 5\cdot x\right) ^{3 - 1}
\end{equation}
\begin{equation}
	C = 
1\cdot 5\cdot x^{5 - 1}\cdot \cos {x^{5}}
\end{equation}
\begin{equation}
	B = 
1\cdot 5\cdot x^{5 - 1}\cdot \frac{\partial}{\partial x}\left( \cos {x^{5}}\right) 
\end{equation}
\begin{equation}
	A = 
\frac{\partial}{\partial x}\left( 1\cdot 5\cdot x^{5 - 1}\right) \cdot \cos {x^{5}}
\end{equation}
Из неравенства Клаузиуса вытекает, что:  \begin{equation}
	\frac{\left( \left( \left( A + B\right) \cdot \cos {x^{5}} + C\right) \cdot \cos {\left( 5\cdot x\right) ^{3}} + D\cdot \frac{\partial}{\partial x}\left( \cos {\left( 5\cdot x\right) ^{3}}\right)  - \frac{\partial}{\partial x}\left( \sin {x^{5}}\cdot -1\cdot E\cdot \sin {\left( 5\cdot x\right) ^{3}}\right) \right) \cdot \left( \cos {\left( 5\cdot x\right) ^{3}}\right) ^{2} - \left( F\cdot \cos {\left( 5\cdot x\right) ^{3}} - \sin {x^{5}}\cdot -1\cdot G\cdot \sin {\left( 5\cdot x\right) ^{3}}\right) \cdot \frac{\partial}{\partial x}\left( \left( \cos {\left( 5\cdot x\right) ^{3}}\right) ^{2}\right) }{\left( \left( \cos {\left( 5\cdot x\right) ^{3}}\right) ^{2}\right) ^{2}}
\end{equation}
Где, 

\begin{equation}
	G = 
\left( 0\cdot x + 5\cdot 1\right) \cdot 3\cdot \left( 5\cdot x\right) ^{3 - 1}
\end{equation}
\begin{equation}
	F = 
1\cdot 5\cdot x^{5 - 1}\cdot \cos {x^{5}}
\end{equation}
\begin{equation}
	E = 
\left( 0\cdot x + 5\cdot 1\right) \cdot 3\cdot \left( 5\cdot x\right) ^{3 - 1}
\end{equation}
\begin{equation}
	D = 
1\cdot 5\cdot x^{5 - 1}\cdot \cos {x^{5}}
\end{equation}
\begin{equation}
	C = 
1\cdot 5\cdot x^{5 - 1}\cdot \frac{\partial}{\partial x}\left( \cos {x^{5}}\right) 
\end{equation}
\begin{equation}
	B = 
1\cdot \frac{\partial}{\partial x}\left( 5\cdot x^{5 - 1}\right) 
\end{equation}
\begin{equation}
	A = 
\frac{\partial}{\partial x}\left( 1\right) \cdot 5\cdot x^{5 - 1}
\end{equation}
По теореме о причёсывании ежа:  \begin{equation}
	\frac{\left( \left( \left( 0\cdot 5\cdot x^{5 - 1} + A\right) \cdot \cos {x^{5}} + B\right) \cdot \cos {\left( 5\cdot x\right) ^{3}} + C\cdot \frac{\partial}{\partial x}\left( \cos {\left( 5\cdot x\right) ^{3}}\right)  - \frac{\partial}{\partial x}\left( \sin {x^{5}}\cdot -1\cdot D\cdot \sin {\left( 5\cdot x\right) ^{3}}\right) \right) \cdot \left( \cos {\left( 5\cdot x\right) ^{3}}\right) ^{2} - \left( E\cdot \cos {\left( 5\cdot x\right) ^{3}} - \sin {x^{5}}\cdot -1\cdot F\cdot \sin {\left( 5\cdot x\right) ^{3}}\right) \cdot \frac{\partial}{\partial x}\left( \left( \cos {\left( 5\cdot x\right) ^{3}}\right) ^{2}\right) }{\left( \left( \cos {\left( 5\cdot x\right) ^{3}}\right) ^{2}\right) ^{2}}
\end{equation}
Где, 

\begin{equation}
	F = 
\left( 0\cdot x + 5\cdot 1\right) \cdot 3\cdot \left( 5\cdot x\right) ^{3 - 1}
\end{equation}
\begin{equation}
	E = 
1\cdot 5\cdot x^{5 - 1}\cdot \cos {x^{5}}
\end{equation}
\begin{equation}
	D = 
\left( 0\cdot x + 5\cdot 1\right) \cdot 3\cdot \left( 5\cdot x\right) ^{3 - 1}
\end{equation}
\begin{equation}
	C = 
1\cdot 5\cdot x^{5 - 1}\cdot \cos {x^{5}}
\end{equation}
\begin{equation}
	B = 
1\cdot 5\cdot x^{5 - 1}\cdot \frac{\partial}{\partial x}\left( \cos {x^{5}}\right) 
\end{equation}
\begin{equation}
	A = 
1\cdot \frac{\partial}{\partial x}\left( 5\cdot x^{5 - 1}\right) 
\end{equation}
Очевидно, что по критерию Сильвестра:  \begin{equation}
	\frac{\left( \left( \left( 0\cdot 5\cdot x^{5 - 1} + 1\cdot \left( A\right) \right) \cdot \cos {x^{5}} + B\right) \cdot \cos {\left( 5\cdot x\right) ^{3}} + C\cdot \frac{\partial}{\partial x}\left( \cos {\left( 5\cdot x\right) ^{3}}\right)  - \frac{\partial}{\partial x}\left( \sin {x^{5}}\cdot -1\cdot D\cdot \sin {\left( 5\cdot x\right) ^{3}}\right) \right) \cdot \left( \cos {\left( 5\cdot x\right) ^{3}}\right) ^{2} - \left( E\cdot \cos {\left( 5\cdot x\right) ^{3}} - \sin {x^{5}}\cdot -1\cdot F\cdot \sin {\left( 5\cdot x\right) ^{3}}\right) \cdot \frac{\partial}{\partial x}\left( \left( \cos {\left( 5\cdot x\right) ^{3}}\right) ^{2}\right) }{\left( \left( \cos {\left( 5\cdot x\right) ^{3}}\right) ^{2}\right) ^{2}}
\end{equation}
Где, 

\begin{equation}
	F = 
\left( 0\cdot x + 5\cdot 1\right) \cdot 3\cdot \left( 5\cdot x\right) ^{3 - 1}
\end{equation}
\begin{equation}
	E = 
1\cdot 5\cdot x^{5 - 1}\cdot \cos {x^{5}}
\end{equation}
\begin{equation}
	D = 
\left( 0\cdot x + 5\cdot 1\right) \cdot 3\cdot \left( 5\cdot x\right) ^{3 - 1}
\end{equation}
\begin{equation}
	C = 
1\cdot 5\cdot x^{5 - 1}\cdot \cos {x^{5}}
\end{equation}
\begin{equation}
	B = 
1\cdot 5\cdot x^{5 - 1}\cdot \frac{\partial}{\partial x}\left( \cos {x^{5}}\right) 
\end{equation}
\begin{equation}
	A = 
\frac{\partial}{\partial x}\left( 5\right) \cdot x^{5 - 1} + 5\cdot \frac{\partial}{\partial x}\left( x^{5 - 1}\right) 
\end{equation}
По теореме об изменении энтропии:  \begin{equation}
	\frac{\left( \left( \left( 0\cdot 5\cdot x^{5 - 1} + A\right) \cdot \cos {x^{5}} + B\right) \cdot \cos {\left( 5\cdot x\right) ^{3}} + C\cdot \frac{\partial}{\partial x}\left( \cos {\left( 5\cdot x\right) ^{3}}\right)  - \frac{\partial}{\partial x}\left( \sin {x^{5}}\cdot -1\cdot D\cdot \sin {\left( 5\cdot x\right) ^{3}}\right) \right) \cdot \left( \cos {\left( 5\cdot x\right) ^{3}}\right) ^{2} - \left( E\cdot \cos {\left( 5\cdot x\right) ^{3}} - \sin {x^{5}}\cdot -1\cdot F\cdot \sin {\left( 5\cdot x\right) ^{3}}\right) \cdot \frac{\partial}{\partial x}\left( \left( \cos {\left( 5\cdot x\right) ^{3}}\right) ^{2}\right) }{\left( \left( \cos {\left( 5\cdot x\right) ^{3}}\right) ^{2}\right) ^{2}}
\end{equation}
Где, 

\begin{equation}
	F = 
\left( 0\cdot x + 5\cdot 1\right) \cdot 3\cdot \left( 5\cdot x\right) ^{3 - 1}
\end{equation}
\begin{equation}
	E = 
1\cdot 5\cdot x^{5 - 1}\cdot \cos {x^{5}}
\end{equation}
\begin{equation}
	D = 
\left( 0\cdot x + 5\cdot 1\right) \cdot 3\cdot \left( 5\cdot x\right) ^{3 - 1}
\end{equation}
\begin{equation}
	C = 
1\cdot 5\cdot x^{5 - 1}\cdot \cos {x^{5}}
\end{equation}
\begin{equation}
	B = 
1\cdot 5\cdot x^{5 - 1}\cdot \frac{\partial}{\partial x}\left( \cos {x^{5}}\right) 
\end{equation}
\begin{equation}
	A = 
1\cdot \left( 0\cdot x^{5 - 1} + 5\cdot \frac{\partial}{\partial x}\left( x^{5 - 1}\right) \right) 
\end{equation}
А Флуктуационно-диссипационная теорема гласит, что:  \begin{equation}
	\frac{\left( \left( \left( 0\cdot 5\cdot x^{5 - 1} + 1\cdot \left( 0\cdot x^{5 - 1} + A\right) \right) \cdot \cos {x^{5}} + B\right) \cdot \cos {\left( 5\cdot x\right) ^{3}} + C\cdot \frac{\partial}{\partial x}\left( \cos {\left( 5\cdot x\right) ^{3}}\right)  - \frac{\partial}{\partial x}\left( \sin {x^{5}}\cdot -1\cdot D\cdot \sin {\left( 5\cdot x\right) ^{3}}\right) \right) \cdot \left( \cos {\left( 5\cdot x\right) ^{3}}\right) ^{2} - \left( E\cdot \cos {\left( 5\cdot x\right) ^{3}} - \sin {x^{5}}\cdot -1\cdot F\cdot \sin {\left( 5\cdot x\right) ^{3}}\right) \cdot \frac{\partial}{\partial x}\left( \left( \cos {\left( 5\cdot x\right) ^{3}}\right) ^{2}\right) }{\left( \left( \cos {\left( 5\cdot x\right) ^{3}}\right) ^{2}\right) ^{2}}
\end{equation}
Где, 

\begin{equation}
	F = 
\left( 0\cdot x + 5\cdot 1\right) \cdot 3\cdot \left( 5\cdot x\right) ^{3 - 1}
\end{equation}
\begin{equation}
	E = 
1\cdot 5\cdot x^{5 - 1}\cdot \cos {x^{5}}
\end{equation}
\begin{equation}
	D = 
\left( 0\cdot x + 5\cdot 1\right) \cdot 3\cdot \left( 5\cdot x\right) ^{3 - 1}
\end{equation}
\begin{equation}
	C = 
1\cdot 5\cdot x^{5 - 1}\cdot \cos {x^{5}}
\end{equation}
\begin{equation}
	B = 
1\cdot 5\cdot x^{5 - 1}\cdot \frac{\partial}{\partial x}\left( \cos {x^{5}}\right) 
\end{equation}
\begin{equation}
	A = 
5\cdot \frac{\partial}{\partial x}\left( x\right) \cdot \left( 5 - 1\right) \cdot x^{5 - 1 - 1}
\end{equation}
Слава Украине, героям слава:  \begin{equation}
	\frac{\left( \left( \left( 0\cdot 5\cdot x^{5 - 1} + 1\cdot \left( 0\cdot x^{5 - 1} + A\right) \right) \cdot \cos {x^{5}} + B\right) \cdot \cos {\left( 5\cdot x\right) ^{3}} + C\cdot \frac{\partial}{\partial x}\left( \cos {\left( 5\cdot x\right) ^{3}}\right)  - \frac{\partial}{\partial x}\left( \sin {x^{5}}\cdot -1\cdot D\cdot \sin {\left( 5\cdot x\right) ^{3}}\right) \right) \cdot \left( \cos {\left( 5\cdot x\right) ^{3}}\right) ^{2} - \left( E\cdot \cos {\left( 5\cdot x\right) ^{3}} - \sin {x^{5}}\cdot -1\cdot F\cdot \sin {\left( 5\cdot x\right) ^{3}}\right) \cdot \frac{\partial}{\partial x}\left( \left( \cos {\left( 5\cdot x\right) ^{3}}\right) ^{2}\right) }{\left( \left( \cos {\left( 5\cdot x\right) ^{3}}\right) ^{2}\right) ^{2}}
\end{equation}
Где, 

\begin{equation}
	F = 
\left( 0\cdot x + 5\cdot 1\right) \cdot 3\cdot \left( 5\cdot x\right) ^{3 - 1}
\end{equation}
\begin{equation}
	E = 
1\cdot 5\cdot x^{5 - 1}\cdot \cos {x^{5}}
\end{equation}
\begin{equation}
	D = 
\left( 0\cdot x + 5\cdot 1\right) \cdot 3\cdot \left( 5\cdot x\right) ^{3 - 1}
\end{equation}
\begin{equation}
	C = 
1\cdot 5\cdot x^{5 - 1}\cdot \cos {x^{5}}
\end{equation}
\begin{equation}
	B = 
1\cdot 5\cdot x^{5 - 1}\cdot \frac{\partial}{\partial x}\left( \cos {x^{5}}\right) 
\end{equation}
\begin{equation}
	A = 
5\cdot 1\cdot \left( 5 - 1\right) \cdot x^{5 - 1 - 1}
\end{equation}
Слава Украине, героям слава:  \begin{equation}
	\frac{\left( \left( \left( 0\cdot 5\cdot x^{5 - 1} + 1\cdot \left( 0\cdot x^{5 - 1} + A\right) \right) \cdot \cos {x^{5}} + 1\cdot 5\cdot x^{5 - 1}\cdot B\right) \cdot \cos {\left( 5\cdot x\right) ^{3}} + C\cdot \frac{\partial}{\partial x}\left( \cos {\left( 5\cdot x\right) ^{3}}\right)  - \frac{\partial}{\partial x}\left( \sin {x^{5}}\cdot -1\cdot D\cdot \sin {\left( 5\cdot x\right) ^{3}}\right) \right) \cdot \left( \cos {\left( 5\cdot x\right) ^{3}}\right) ^{2} - \left( E\cdot \cos {\left( 5\cdot x\right) ^{3}} - \sin {x^{5}}\cdot -1\cdot F\cdot \sin {\left( 5\cdot x\right) ^{3}}\right) \cdot \frac{\partial}{\partial x}\left( \left( \cos {\left( 5\cdot x\right) ^{3}}\right) ^{2}\right) }{\left( \left( \cos {\left( 5\cdot x\right) ^{3}}\right) ^{2}\right) ^{2}}
\end{equation}
Где, 

\begin{equation}
	F = 
\left( 0\cdot x + 5\cdot 1\right) \cdot 3\cdot \left( 5\cdot x\right) ^{3 - 1}
\end{equation}
\begin{equation}
	E = 
1\cdot 5\cdot x^{5 - 1}\cdot \cos {x^{5}}
\end{equation}
\begin{equation}
	D = 
\left( 0\cdot x + 5\cdot 1\right) \cdot 3\cdot \left( 5\cdot x\right) ^{3 - 1}
\end{equation}
\begin{equation}
	C = 
1\cdot 5\cdot x^{5 - 1}\cdot \cos {x^{5}}
\end{equation}
\begin{equation}
	B = 
-1\cdot \frac{\partial}{\partial x}\left( x^{5}\right) \cdot \sin {x^{5}}
\end{equation}
\begin{equation}
	A = 
5\cdot 1\cdot \left( 5 - 1\right) \cdot x^{5 - 1 - 1}
\end{equation}
Очевидно, что по критерию Сильвестра:  \begin{equation}
	\frac{\left( \left( \left( 0\cdot 5\cdot x^{5 - 1} + 1\cdot \left( 0\cdot x^{5 - 1} + A\right) \right) \cdot \cos {x^{5}} + 1\cdot 5\cdot x^{5 - 1}\cdot B\right) \cdot \cos {\left( 5\cdot x\right) ^{3}} + C\cdot \frac{\partial}{\partial x}\left( \cos {\left( 5\cdot x\right) ^{3}}\right)  - \frac{\partial}{\partial x}\left( \sin {x^{5}}\cdot -1\cdot D\cdot \sin {\left( 5\cdot x\right) ^{3}}\right) \right) \cdot \left( \cos {\left( 5\cdot x\right) ^{3}}\right) ^{2} - \left( E\cdot \cos {\left( 5\cdot x\right) ^{3}} - \sin {x^{5}}\cdot -1\cdot F\cdot \sin {\left( 5\cdot x\right) ^{3}}\right) \cdot \frac{\partial}{\partial x}\left( \left( \cos {\left( 5\cdot x\right) ^{3}}\right) ^{2}\right) }{\left( \left( \cos {\left( 5\cdot x\right) ^{3}}\right) ^{2}\right) ^{2}}
\end{equation}
Где, 

\begin{equation}
	F = 
\left( 0\cdot x + 5\cdot 1\right) \cdot 3\cdot \left( 5\cdot x\right) ^{3 - 1}
\end{equation}
\begin{equation}
	E = 
1\cdot 5\cdot x^{5 - 1}\cdot \cos {x^{5}}
\end{equation}
\begin{equation}
	D = 
\left( 0\cdot x + 5\cdot 1\right) \cdot 3\cdot \left( 5\cdot x\right) ^{3 - 1}
\end{equation}
\begin{equation}
	C = 
1\cdot 5\cdot x^{5 - 1}\cdot \cos {x^{5}}
\end{equation}
\begin{equation}
	B = 
-1\cdot \frac{\partial}{\partial x}\left( x\right) \cdot 5\cdot x^{5 - 1}\cdot \sin {x^{5}}
\end{equation}
\begin{equation}
	A = 
5\cdot 1\cdot \left( 5 - 1\right) \cdot x^{5 - 1 - 1}
\end{equation}
А вот если вспомнить теорему Пенроуза — Хокинга о сингулярности, то:  \begin{equation}
	\frac{\left( \left( \left( 0\cdot 5\cdot x^{5 - 1} + 1\cdot \left( 0\cdot x^{5 - 1} + A\right) \right) \cdot \cos {x^{5}} + 1\cdot 5\cdot x^{5 - 1}\cdot B\right) \cdot \cos {\left( 5\cdot x\right) ^{3}} + C\cdot \frac{\partial}{\partial x}\left( \cos {\left( 5\cdot x\right) ^{3}}\right)  - \frac{\partial}{\partial x}\left( \sin {x^{5}}\cdot -1\cdot D\cdot \sin {\left( 5\cdot x\right) ^{3}}\right) \right) \cdot \left( \cos {\left( 5\cdot x\right) ^{3}}\right) ^{2} - \left( E\cdot \cos {\left( 5\cdot x\right) ^{3}} - \sin {x^{5}}\cdot -1\cdot F\cdot \sin {\left( 5\cdot x\right) ^{3}}\right) \cdot \frac{\partial}{\partial x}\left( \left( \cos {\left( 5\cdot x\right) ^{3}}\right) ^{2}\right) }{\left( \left( \cos {\left( 5\cdot x\right) ^{3}}\right) ^{2}\right) ^{2}}
\end{equation}
Где, 

\begin{equation}
	F = 
\left( 0\cdot x + 5\cdot 1\right) \cdot 3\cdot \left( 5\cdot x\right) ^{3 - 1}
\end{equation}
\begin{equation}
	E = 
1\cdot 5\cdot x^{5 - 1}\cdot \cos {x^{5}}
\end{equation}
\begin{equation}
	D = 
\left( 0\cdot x + 5\cdot 1\right) \cdot 3\cdot \left( 5\cdot x\right) ^{3 - 1}
\end{equation}
\begin{equation}
	C = 
1\cdot 5\cdot x^{5 - 1}\cdot \cos {x^{5}}
\end{equation}
\begin{equation}
	B = 
-1\cdot 1\cdot 5\cdot x^{5 - 1}\cdot \sin {x^{5}}
\end{equation}
\begin{equation}
	A = 
5\cdot 1\cdot \left( 5 - 1\right) \cdot x^{5 - 1 - 1}
\end{equation}
Очевидно, что по критерию Сильвестра:  \begin{equation}
	\frac{\left( \left( \left( 0\cdot 5\cdot x^{5 - 1} + 1\cdot \left( 0\cdot x^{5 - 1} + A\right) \right) \cdot \cos {x^{5}} + 1\cdot 5\cdot x^{5 - 1}\cdot B\right) \cdot \cos {\left( 5\cdot x\right) ^{3}} + C\cdot D - \frac{\partial}{\partial x}\left( \sin {x^{5}}\cdot -1\cdot E\cdot \sin {\left( 5\cdot x\right) ^{3}}\right) \right) \cdot \left( \cos {\left( 5\cdot x\right) ^{3}}\right) ^{2} - \left( F\cdot \cos {\left( 5\cdot x\right) ^{3}} - \sin {x^{5}}\cdot -1\cdot G\cdot \sin {\left( 5\cdot x\right) ^{3}}\right) \cdot \frac{\partial}{\partial x}\left( \left( \cos {\left( 5\cdot x\right) ^{3}}\right) ^{2}\right) }{\left( \left( \cos {\left( 5\cdot x\right) ^{3}}\right) ^{2}\right) ^{2}}
\end{equation}
Где, 

\begin{equation}
	G = 
\left( 0\cdot x + 5\cdot 1\right) \cdot 3\cdot \left( 5\cdot x\right) ^{3 - 1}
\end{equation}
\begin{equation}
	F = 
1\cdot 5\cdot x^{5 - 1}\cdot \cos {x^{5}}
\end{equation}
\begin{equation}
	E = 
\left( 0\cdot x + 5\cdot 1\right) \cdot 3\cdot \left( 5\cdot x\right) ^{3 - 1}
\end{equation}
\begin{equation}
	D = 
-1\cdot \frac{\partial}{\partial x}\left( \left( 5\cdot x\right) ^{3}\right) \cdot \sin {\left( 5\cdot x\right) ^{3}}
\end{equation}
\begin{equation}
	C = 
1\cdot 5\cdot x^{5 - 1}\cdot \cos {x^{5}}
\end{equation}
\begin{equation}
	B = 
-1\cdot 1\cdot 5\cdot x^{5 - 1}\cdot \sin {x^{5}}
\end{equation}
\begin{equation}
	A = 
5\cdot 1\cdot \left( 5 - 1\right) \cdot x^{5 - 1 - 1}
\end{equation}
А вот если вспомнить теорему Пенроуза — Хокинга о сингулярности, то:  \begin{equation}
	\frac{\left( \left( \left( 0\cdot 5\cdot x^{5 - 1} + 1\cdot \left( 0\cdot x^{5 - 1} + A\right) \right) \cdot \cos {x^{5}} + 1\cdot 5\cdot x^{5 - 1}\cdot B\right) \cdot \cos {\left( 5\cdot x\right) ^{3}} + C\cdot -1\cdot D\cdot \sin {\left( 5\cdot x\right) ^{3}} - \frac{\partial}{\partial x}\left( \sin {x^{5}}\cdot -1\cdot E\cdot \sin {\left( 5\cdot x\right) ^{3}}\right) \right) \cdot \left( \cos {\left( 5\cdot x\right) ^{3}}\right) ^{2} - \left( F\cdot \cos {\left( 5\cdot x\right) ^{3}} - \sin {x^{5}}\cdot -1\cdot G\cdot \sin {\left( 5\cdot x\right) ^{3}}\right) \cdot \frac{\partial}{\partial x}\left( \left( \cos {\left( 5\cdot x\right) ^{3}}\right) ^{2}\right) }{\left( \left( \cos {\left( 5\cdot x\right) ^{3}}\right) ^{2}\right) ^{2}}
\end{equation}
Где, 

\begin{equation}
	G = 
\left( 0\cdot x + 5\cdot 1\right) \cdot 3\cdot \left( 5\cdot x\right) ^{3 - 1}
\end{equation}
\begin{equation}
	F = 
1\cdot 5\cdot x^{5 - 1}\cdot \cos {x^{5}}
\end{equation}
\begin{equation}
	E = 
\left( 0\cdot x + 5\cdot 1\right) \cdot 3\cdot \left( 5\cdot x\right) ^{3 - 1}
\end{equation}
\begin{equation}
	D = 
\frac{\partial}{\partial x}\left( 5\cdot x\right) \cdot 3\cdot \left( 5\cdot x\right) ^{3 - 1}
\end{equation}
\begin{equation}
	C = 
1\cdot 5\cdot x^{5 - 1}\cdot \cos {x^{5}}
\end{equation}
\begin{equation}
	B = 
-1\cdot 1\cdot 5\cdot x^{5 - 1}\cdot \sin {x^{5}}
\end{equation}
\begin{equation}
	A = 
5\cdot 1\cdot \left( 5 - 1\right) \cdot x^{5 - 1 - 1}
\end{equation}
А вот если вспомнить теорему Пенроуза — Хокинга о сингулярности, то:  \begin{equation}
	\frac{\left( \left( \left( 0\cdot 5\cdot x^{5 - 1} + 1\cdot \left( 0\cdot x^{5 - 1} + A\right) \right) \cdot \cos {x^{5}} + 1\cdot 5\cdot x^{5 - 1}\cdot B\right) \cdot \cos {\left( 5\cdot x\right) ^{3}} + C\cdot -1\cdot \left( D\right) \cdot 3\cdot \left( 5\cdot x\right) ^{3 - 1}\cdot \sin {\left( 5\cdot x\right) ^{3}} - \frac{\partial}{\partial x}\left( \sin {x^{5}}\cdot -1\cdot E\cdot \sin {\left( 5\cdot x\right) ^{3}}\right) \right) \cdot \left( \cos {\left( 5\cdot x\right) ^{3}}\right) ^{2} - \left( F\cdot \cos {\left( 5\cdot x\right) ^{3}} - \sin {x^{5}}\cdot -1\cdot G\cdot \sin {\left( 5\cdot x\right) ^{3}}\right) \cdot \frac{\partial}{\partial x}\left( \left( \cos {\left( 5\cdot x\right) ^{3}}\right) ^{2}\right) }{\left( \left( \cos {\left( 5\cdot x\right) ^{3}}\right) ^{2}\right) ^{2}}
\end{equation}
Где, 

\begin{equation}
	G = 
\left( 0\cdot x + 5\cdot 1\right) \cdot 3\cdot \left( 5\cdot x\right) ^{3 - 1}
\end{equation}
\begin{equation}
	F = 
1\cdot 5\cdot x^{5 - 1}\cdot \cos {x^{5}}
\end{equation}
\begin{equation}
	E = 
\left( 0\cdot x + 5\cdot 1\right) \cdot 3\cdot \left( 5\cdot x\right) ^{3 - 1}
\end{equation}
\begin{equation}
	D = 
\frac{\partial}{\partial x}\left( 5\right) \cdot x + 5\cdot \frac{\partial}{\partial x}\left( x\right) 
\end{equation}
\begin{equation}
	C = 
1\cdot 5\cdot x^{5 - 1}\cdot \cos {x^{5}}
\end{equation}
\begin{equation}
	B = 
-1\cdot 1\cdot 5\cdot x^{5 - 1}\cdot \sin {x^{5}}
\end{equation}
\begin{equation}
	A = 
5\cdot 1\cdot \left( 5 - 1\right) \cdot x^{5 - 1 - 1}
\end{equation}
А Флуктуационно-диссипационная теорема гласит, что:  \begin{equation}
	\frac{\left( \left( \left( 0\cdot 5\cdot x^{5 - 1} + 1\cdot \left( 0\cdot x^{5 - 1} + A\right) \right) \cdot \cos {x^{5}} + 1\cdot 5\cdot x^{5 - 1}\cdot B\right) \cdot \cos {\left( 5\cdot x\right) ^{3}} + C\cdot -1\cdot D\cdot \sin {\left( 5\cdot x\right) ^{3}} - \frac{\partial}{\partial x}\left( \sin {x^{5}}\cdot -1\cdot E\cdot \sin {\left( 5\cdot x\right) ^{3}}\right) \right) \cdot \left( \cos {\left( 5\cdot x\right) ^{3}}\right) ^{2} - \left( F\cdot \cos {\left( 5\cdot x\right) ^{3}} - \sin {x^{5}}\cdot -1\cdot G\cdot \sin {\left( 5\cdot x\right) ^{3}}\right) \cdot \frac{\partial}{\partial x}\left( \left( \cos {\left( 5\cdot x\right) ^{3}}\right) ^{2}\right) }{\left( \left( \cos {\left( 5\cdot x\right) ^{3}}\right) ^{2}\right) ^{2}}
\end{equation}
Где, 

\begin{equation}
	G = 
\left( 0\cdot x + 5\cdot 1\right) \cdot 3\cdot \left( 5\cdot x\right) ^{3 - 1}
\end{equation}
\begin{equation}
	F = 
1\cdot 5\cdot x^{5 - 1}\cdot \cos {x^{5}}
\end{equation}
\begin{equation}
	E = 
\left( 0\cdot x + 5\cdot 1\right) \cdot 3\cdot \left( 5\cdot x\right) ^{3 - 1}
\end{equation}
\begin{equation}
	D = 
\left( 0\cdot x + 5\cdot \frac{\partial}{\partial x}\left( x\right) \right) \cdot 3\cdot \left( 5\cdot x\right) ^{3 - 1}
\end{equation}
\begin{equation}
	C = 
1\cdot 5\cdot x^{5 - 1}\cdot \cos {x^{5}}
\end{equation}
\begin{equation}
	B = 
-1\cdot 1\cdot 5\cdot x^{5 - 1}\cdot \sin {x^{5}}
\end{equation}
\begin{equation}
	A = 
5\cdot 1\cdot \left( 5 - 1\right) \cdot x^{5 - 1 - 1}
\end{equation}
А Флуктуационно-диссипационная теорема гласит, что:  \begin{equation}
	\frac{\left( \left( \left( 0\cdot 5\cdot x^{5 - 1} + 1\cdot \left( 0\cdot x^{5 - 1} + A\right) \right) \cdot \cos {x^{5}} + 1\cdot 5\cdot x^{5 - 1}\cdot B\right) \cdot \cos {\left( 5\cdot x\right) ^{3}} + C\cdot -1\cdot D\cdot \sin {\left( 5\cdot x\right) ^{3}} - \frac{\partial}{\partial x}\left( \sin {x^{5}}\cdot -1\cdot E\cdot \sin {\left( 5\cdot x\right) ^{3}}\right) \right) \cdot \left( \cos {\left( 5\cdot x\right) ^{3}}\right) ^{2} - \left( F\cdot \cos {\left( 5\cdot x\right) ^{3}} - \sin {x^{5}}\cdot -1\cdot G\cdot \sin {\left( 5\cdot x\right) ^{3}}\right) \cdot \frac{\partial}{\partial x}\left( \left( \cos {\left( 5\cdot x\right) ^{3}}\right) ^{2}\right) }{\left( \left( \cos {\left( 5\cdot x\right) ^{3}}\right) ^{2}\right) ^{2}}
\end{equation}
Где, 

\begin{equation}
	G = 
\left( 0\cdot x + 5\cdot 1\right) \cdot 3\cdot \left( 5\cdot x\right) ^{3 - 1}
\end{equation}
\begin{equation}
	F = 
1\cdot 5\cdot x^{5 - 1}\cdot \cos {x^{5}}
\end{equation}
\begin{equation}
	E = 
\left( 0\cdot x + 5\cdot 1\right) \cdot 3\cdot \left( 5\cdot x\right) ^{3 - 1}
\end{equation}
\begin{equation}
	D = 
\left( 0\cdot x + 5\cdot 1\right) \cdot 3\cdot \left( 5\cdot x\right) ^{3 - 1}
\end{equation}
\begin{equation}
	C = 
1\cdot 5\cdot x^{5 - 1}\cdot \cos {x^{5}}
\end{equation}
\begin{equation}
	B = 
-1\cdot 1\cdot 5\cdot x^{5 - 1}\cdot \sin {x^{5}}
\end{equation}
\begin{equation}
	A = 
5\cdot 1\cdot \left( 5 - 1\right) \cdot x^{5 - 1 - 1}
\end{equation}
По теореме об изменении энтропии:  \begin{equation}
	\frac{\left( \left( \left( 0\cdot 5\cdot x^{5 - 1} + 1\cdot \left( 0\cdot x^{5 - 1} + A\right) \right) \cdot \cos {x^{5}} + 1\cdot 5\cdot x^{5 - 1}\cdot B\right) \cdot \cos {\left( 5\cdot x\right) ^{3}} + C\cdot -1\cdot D\cdot \sin {\left( 5\cdot x\right) ^{3}} - \frac{\partial}{\partial x}\left( \sin {x^{5}}\right) \cdot -1\cdot E\cdot \sin {\left( 5\cdot x\right) ^{3}} + \sin {x^{5}}\cdot \frac{\partial}{\partial x}\left( -1\cdot F\cdot \sin {\left( 5\cdot x\right) ^{3}}\right) \right) \cdot \left( \cos {\left( 5\cdot x\right) ^{3}}\right) ^{2} - \left( G\cdot \cos {\left( 5\cdot x\right) ^{3}} - \sin {x^{5}}\cdot -1\cdot H\cdot \sin {\left( 5\cdot x\right) ^{3}}\right) \cdot \frac{\partial}{\partial x}\left( \left( \cos {\left( 5\cdot x\right) ^{3}}\right) ^{2}\right) }{\left( \left( \cos {\left( 5\cdot x\right) ^{3}}\right) ^{2}\right) ^{2}}
\end{equation}
Где, 

\begin{equation}
	H = 
\left( 0\cdot x + 5\cdot 1\right) \cdot 3\cdot \left( 5\cdot x\right) ^{3 - 1}
\end{equation}
\begin{equation}
	G = 
1\cdot 5\cdot x^{5 - 1}\cdot \cos {x^{5}}
\end{equation}
\begin{equation}
	F = 
\left( 0\cdot x + 5\cdot 1\right) \cdot 3\cdot \left( 5\cdot x\right) ^{3 - 1}
\end{equation}
\begin{equation}
	E = 
\left( 0\cdot x + 5\cdot 1\right) \cdot 3\cdot \left( 5\cdot x\right) ^{3 - 1}
\end{equation}
\begin{equation}
	D = 
\left( 0\cdot x + 5\cdot 1\right) \cdot 3\cdot \left( 5\cdot x\right) ^{3 - 1}
\end{equation}
\begin{equation}
	C = 
1\cdot 5\cdot x^{5 - 1}\cdot \cos {x^{5}}
\end{equation}
\begin{equation}
	B = 
-1\cdot 1\cdot 5\cdot x^{5 - 1}\cdot \sin {x^{5}}
\end{equation}
\begin{equation}
	A = 
5\cdot 1\cdot \left( 5 - 1\right) \cdot x^{5 - 1 - 1}
\end{equation}
Слава Украине, героям слава:  \begin{equation}
	\frac{\left( \left( \left( 0\cdot 5\cdot x^{5 - 1} + 1\cdot \left( 0\cdot x^{5 - 1} + A\right) \right) \cdot \cos {x^{5}} + 1\cdot 5\cdot x^{5 - 1}\cdot B\right) \cdot \cos {\left( 5\cdot x\right) ^{3}} + C\cdot -1\cdot D\cdot \sin {\left( 5\cdot x\right) ^{3}} - \frac{\partial}{\partial x}\left( x^{5}\right) \cdot \cos {x^{5}}\cdot -1\cdot E\cdot \sin {\left( 5\cdot x\right) ^{3}} + \sin {x^{5}}\cdot \frac{\partial}{\partial x}\left( -1\cdot F\cdot \sin {\left( 5\cdot x\right) ^{3}}\right) \right) \cdot \left( \cos {\left( 5\cdot x\right) ^{3}}\right) ^{2} - \left( G\cdot \cos {\left( 5\cdot x\right) ^{3}} - \sin {x^{5}}\cdot -1\cdot H\cdot \sin {\left( 5\cdot x\right) ^{3}}\right) \cdot \frac{\partial}{\partial x}\left( \left( \cos {\left( 5\cdot x\right) ^{3}}\right) ^{2}\right) }{\left( \left( \cos {\left( 5\cdot x\right) ^{3}}\right) ^{2}\right) ^{2}}
\end{equation}
Где, 

\begin{equation}
	H = 
\left( 0\cdot x + 5\cdot 1\right) \cdot 3\cdot \left( 5\cdot x\right) ^{3 - 1}
\end{equation}
\begin{equation}
	G = 
1\cdot 5\cdot x^{5 - 1}\cdot \cos {x^{5}}
\end{equation}
\begin{equation}
	F = 
\left( 0\cdot x + 5\cdot 1\right) \cdot 3\cdot \left( 5\cdot x\right) ^{3 - 1}
\end{equation}
\begin{equation}
	E = 
\left( 0\cdot x + 5\cdot 1\right) \cdot 3\cdot \left( 5\cdot x\right) ^{3 - 1}
\end{equation}
\begin{equation}
	D = 
\left( 0\cdot x + 5\cdot 1\right) \cdot 3\cdot \left( 5\cdot x\right) ^{3 - 1}
\end{equation}
\begin{equation}
	C = 
1\cdot 5\cdot x^{5 - 1}\cdot \cos {x^{5}}
\end{equation}
\begin{equation}
	B = 
-1\cdot 1\cdot 5\cdot x^{5 - 1}\cdot \sin {x^{5}}
\end{equation}
\begin{equation}
	A = 
5\cdot 1\cdot \left( 5 - 1\right) \cdot x^{5 - 1 - 1}
\end{equation}
Из неравенства Клаузиуса вытекает, что:  \begin{equation}
	\frac{\left( \left( \left( 0\cdot 5\cdot x^{5 - 1} + 1\cdot \left( 0\cdot x^{5 - 1} + A\right) \right) \cdot \cos {x^{5}} + 1\cdot 5\cdot x^{5 - 1}\cdot B\right) \cdot \cos {\left( 5\cdot x\right) ^{3}} + C\cdot -1\cdot D\cdot \sin {\left( 5\cdot x\right) ^{3}} - E\cdot -1\cdot F\cdot \sin {\left( 5\cdot x\right) ^{3}} + \sin {x^{5}}\cdot \frac{\partial}{\partial x}\left( -1\cdot G\cdot \sin {\left( 5\cdot x\right) ^{3}}\right) \right) \cdot \left( \cos {\left( 5\cdot x\right) ^{3}}\right) ^{2} - \left( H\cdot \cos {\left( 5\cdot x\right) ^{3}} - \sin {x^{5}}\cdot -1\cdot I\cdot \sin {\left( 5\cdot x\right) ^{3}}\right) \cdot \frac{\partial}{\partial x}\left( \left( \cos {\left( 5\cdot x\right) ^{3}}\right) ^{2}\right) }{\left( \left( \cos {\left( 5\cdot x\right) ^{3}}\right) ^{2}\right) ^{2}}
\end{equation}
Где, 

\begin{equation}
	I = 
\left( 0\cdot x + 5\cdot 1\right) \cdot 3\cdot \left( 5\cdot x\right) ^{3 - 1}
\end{equation}
\begin{equation}
	H = 
1\cdot 5\cdot x^{5 - 1}\cdot \cos {x^{5}}
\end{equation}
\begin{equation}
	G = 
\left( 0\cdot x + 5\cdot 1\right) \cdot 3\cdot \left( 5\cdot x\right) ^{3 - 1}
\end{equation}
\begin{equation}
	F = 
\left( 0\cdot x + 5\cdot 1\right) \cdot 3\cdot \left( 5\cdot x\right) ^{3 - 1}
\end{equation}
\begin{equation}
	E = 
\frac{\partial}{\partial x}\left( x\right) \cdot 5\cdot x^{5 - 1}\cdot \cos {x^{5}}
\end{equation}
\begin{equation}
	D = 
\left( 0\cdot x + 5\cdot 1\right) \cdot 3\cdot \left( 5\cdot x\right) ^{3 - 1}
\end{equation}
\begin{equation}
	C = 
1\cdot 5\cdot x^{5 - 1}\cdot \cos {x^{5}}
\end{equation}
\begin{equation}
	B = 
-1\cdot 1\cdot 5\cdot x^{5 - 1}\cdot \sin {x^{5}}
\end{equation}
\begin{equation}
	A = 
5\cdot 1\cdot \left( 5 - 1\right) \cdot x^{5 - 1 - 1}
\end{equation}
Очевидно, что по критерию Сильвестра:  \begin{equation}
	\frac{\left( \left( \left( 0\cdot 5\cdot x^{5 - 1} + 1\cdot \left( 0\cdot x^{5 - 1} + A\right) \right) \cdot \cos {x^{5}} + 1\cdot 5\cdot x^{5 - 1}\cdot B\right) \cdot \cos {\left( 5\cdot x\right) ^{3}} + C\cdot -1\cdot D\cdot \sin {\left( 5\cdot x\right) ^{3}} - E\cdot -1\cdot F\cdot \sin {\left( 5\cdot x\right) ^{3}} + \sin {x^{5}}\cdot \frac{\partial}{\partial x}\left( -1\cdot G\cdot \sin {\left( 5\cdot x\right) ^{3}}\right) \right) \cdot \left( \cos {\left( 5\cdot x\right) ^{3}}\right) ^{2} - \left( H\cdot \cos {\left( 5\cdot x\right) ^{3}} - \sin {x^{5}}\cdot -1\cdot I\cdot \sin {\left( 5\cdot x\right) ^{3}}\right) \cdot \frac{\partial}{\partial x}\left( \left( \cos {\left( 5\cdot x\right) ^{3}}\right) ^{2}\right) }{\left( \left( \cos {\left( 5\cdot x\right) ^{3}}\right) ^{2}\right) ^{2}}
\end{equation}
Где, 

\begin{equation}
	I = 
\left( 0\cdot x + 5\cdot 1\right) \cdot 3\cdot \left( 5\cdot x\right) ^{3 - 1}
\end{equation}
\begin{equation}
	H = 
1\cdot 5\cdot x^{5 - 1}\cdot \cos {x^{5}}
\end{equation}
\begin{equation}
	G = 
\left( 0\cdot x + 5\cdot 1\right) \cdot 3\cdot \left( 5\cdot x\right) ^{3 - 1}
\end{equation}
\begin{equation}
	F = 
\left( 0\cdot x + 5\cdot 1\right) \cdot 3\cdot \left( 5\cdot x\right) ^{3 - 1}
\end{equation}
\begin{equation}
	E = 
1\cdot 5\cdot x^{5 - 1}\cdot \cos {x^{5}}
\end{equation}
\begin{equation}
	D = 
\left( 0\cdot x + 5\cdot 1\right) \cdot 3\cdot \left( 5\cdot x\right) ^{3 - 1}
\end{equation}
\begin{equation}
	C = 
1\cdot 5\cdot x^{5 - 1}\cdot \cos {x^{5}}
\end{equation}
\begin{equation}
	B = 
-1\cdot 1\cdot 5\cdot x^{5 - 1}\cdot \sin {x^{5}}
\end{equation}
\begin{equation}
	A = 
5\cdot 1\cdot \left( 5 - 1\right) \cdot x^{5 - 1 - 1}
\end{equation}
Гаусс еще в \RomanNumeralCaps{14} веке посчитал, что:  \begin{equation}
	\frac{\left( \left( \left( 0\cdot 5\cdot x^{5 - 1} + 1\cdot \left( 0\cdot x^{5 - 1} + A\right) \right) \cdot \cos {x^{5}} + 1\cdot 5\cdot x^{5 - 1}\cdot B\right) \cdot \cos {\left( 5\cdot x\right) ^{3}} + C\cdot -1\cdot D\cdot \sin {\left( 5\cdot x\right) ^{3}} - E\cdot -1\cdot F\cdot \sin {\left( 5\cdot x\right) ^{3}} + \sin {x^{5}}\cdot \left( \frac{\partial}{\partial x}\left( -1\right) \cdot G\cdot \sin {\left( 5\cdot x\right) ^{3}} + -1\cdot \frac{\partial}{\partial x}\left( H\cdot \sin {\left( 5\cdot x\right) ^{3}}\right) \right) \right) \cdot \left( \cos {\left( 5\cdot x\right) ^{3}}\right) ^{2} - \left( I\cdot \cos {\left( 5\cdot x\right) ^{3}} - \sin {x^{5}}\cdot -1\cdot J\cdot \sin {\left( 5\cdot x\right) ^{3}}\right) \cdot \frac{\partial}{\partial x}\left( \left( \cos {\left( 5\cdot x\right) ^{3}}\right) ^{2}\right) }{\left( \left( \cos {\left( 5\cdot x\right) ^{3}}\right) ^{2}\right) ^{2}}
\end{equation}
Где, 

\begin{equation}
	J = 
\left( 0\cdot x + 5\cdot 1\right) \cdot 3\cdot \left( 5\cdot x\right) ^{3 - 1}
\end{equation}
\begin{equation}
	I = 
1\cdot 5\cdot x^{5 - 1}\cdot \cos {x^{5}}
\end{equation}
\begin{equation}
	H = 
\left( 0\cdot x + 5\cdot 1\right) \cdot 3\cdot \left( 5\cdot x\right) ^{3 - 1}
\end{equation}
\begin{equation}
	G = 
\left( 0\cdot x + 5\cdot 1\right) \cdot 3\cdot \left( 5\cdot x\right) ^{3 - 1}
\end{equation}
\begin{equation}
	F = 
\left( 0\cdot x + 5\cdot 1\right) \cdot 3\cdot \left( 5\cdot x\right) ^{3 - 1}
\end{equation}
\begin{equation}
	E = 
1\cdot 5\cdot x^{5 - 1}\cdot \cos {x^{5}}
\end{equation}
\begin{equation}
	D = 
\left( 0\cdot x + 5\cdot 1\right) \cdot 3\cdot \left( 5\cdot x\right) ^{3 - 1}
\end{equation}
\begin{equation}
	C = 
1\cdot 5\cdot x^{5 - 1}\cdot \cos {x^{5}}
\end{equation}
\begin{equation}
	B = 
-1\cdot 1\cdot 5\cdot x^{5 - 1}\cdot \sin {x^{5}}
\end{equation}
\begin{equation}
	A = 
5\cdot 1\cdot \left( 5 - 1\right) \cdot x^{5 - 1 - 1}
\end{equation}
А Флуктуационно-диссипационная теорема гласит, что:  \begin{equation}
	\frac{\left( \left( \left( 0\cdot 5\cdot x^{5 - 1} + 1\cdot \left( 0\cdot x^{5 - 1} + A\right) \right) \cdot \cos {x^{5}} + 1\cdot 5\cdot x^{5 - 1}\cdot B\right) \cdot \cos {\left( 5\cdot x\right) ^{3}} + C\cdot -1\cdot D\cdot \sin {\left( 5\cdot x\right) ^{3}} - E\cdot -1\cdot F\cdot \sin {\left( 5\cdot x\right) ^{3}} + \sin {x^{5}}\cdot \left( 0\cdot G\cdot \sin {\left( 5\cdot x\right) ^{3}} + -1\cdot \frac{\partial}{\partial x}\left( H\cdot \sin {\left( 5\cdot x\right) ^{3}}\right) \right) \right) \cdot \left( \cos {\left( 5\cdot x\right) ^{3}}\right) ^{2} - \left( I\cdot \cos {\left( 5\cdot x\right) ^{3}} - \sin {x^{5}}\cdot -1\cdot J\cdot \sin {\left( 5\cdot x\right) ^{3}}\right) \cdot \frac{\partial}{\partial x}\left( \left( \cos {\left( 5\cdot x\right) ^{3}}\right) ^{2}\right) }{\left( \left( \cos {\left( 5\cdot x\right) ^{3}}\right) ^{2}\right) ^{2}}
\end{equation}
Где, 

\begin{equation}
	J = 
\left( 0\cdot x + 5\cdot 1\right) \cdot 3\cdot \left( 5\cdot x\right) ^{3 - 1}
\end{equation}
\begin{equation}
	I = 
1\cdot 5\cdot x^{5 - 1}\cdot \cos {x^{5}}
\end{equation}
\begin{equation}
	H = 
\left( 0\cdot x + 5\cdot 1\right) \cdot 3\cdot \left( 5\cdot x\right) ^{3 - 1}
\end{equation}
\begin{equation}
	G = 
\left( 0\cdot x + 5\cdot 1\right) \cdot 3\cdot \left( 5\cdot x\right) ^{3 - 1}
\end{equation}
\begin{equation}
	F = 
\left( 0\cdot x + 5\cdot 1\right) \cdot 3\cdot \left( 5\cdot x\right) ^{3 - 1}
\end{equation}
\begin{equation}
	E = 
1\cdot 5\cdot x^{5 - 1}\cdot \cos {x^{5}}
\end{equation}
\begin{equation}
	D = 
\left( 0\cdot x + 5\cdot 1\right) \cdot 3\cdot \left( 5\cdot x\right) ^{3 - 1}
\end{equation}
\begin{equation}
	C = 
1\cdot 5\cdot x^{5 - 1}\cdot \cos {x^{5}}
\end{equation}
\begin{equation}
	B = 
-1\cdot 1\cdot 5\cdot x^{5 - 1}\cdot \sin {x^{5}}
\end{equation}
\begin{equation}
	A = 
5\cdot 1\cdot \left( 5 - 1\right) \cdot x^{5 - 1 - 1}
\end{equation}
Как известно, по теореме Пифагора:  \begin{equation}
	\frac{\left( \left( \left( 0\cdot 5\cdot x^{5 - 1} + 1\cdot \left( 0\cdot x^{5 - 1} + A\right) \right) \cdot \cos {x^{5}} + 1\cdot 5\cdot x^{5 - 1}\cdot B\right) \cdot \cos {\left( 5\cdot x\right) ^{3}} + C\cdot -1\cdot D\cdot \sin {\left( 5\cdot x\right) ^{3}} - E\cdot -1\cdot F\cdot \sin {\left( 5\cdot x\right) ^{3}} + \sin {x^{5}}\cdot \left( 0\cdot G\cdot \sin {\left( 5\cdot x\right) ^{3}} + -1\cdot \left( H\cdot \sin {\left( 5\cdot x\right) ^{3}} + I\cdot \frac{\partial}{\partial x}\left( \sin {\left( 5\cdot x\right) ^{3}}\right) \right) \right) \right) \cdot \left( \cos {\left( 5\cdot x\right) ^{3}}\right) ^{2} - \left( J\cdot \cos {\left( 5\cdot x\right) ^{3}} - \sin {x^{5}}\cdot -1\cdot K\cdot \sin {\left( 5\cdot x\right) ^{3}}\right) \cdot \frac{\partial}{\partial x}\left( \left( \cos {\left( 5\cdot x\right) ^{3}}\right) ^{2}\right) }{\left( \left( \cos {\left( 5\cdot x\right) ^{3}}\right) ^{2}\right) ^{2}}
\end{equation}
Где, 

\begin{equation}
	K = 
\left( 0\cdot x + 5\cdot 1\right) \cdot 3\cdot \left( 5\cdot x\right) ^{3 - 1}
\end{equation}
\begin{equation}
	J = 
1\cdot 5\cdot x^{5 - 1}\cdot \cos {x^{5}}
\end{equation}
\begin{equation}
	I = 
\left( 0\cdot x + 5\cdot 1\right) \cdot 3\cdot \left( 5\cdot x\right) ^{3 - 1}
\end{equation}
\begin{equation}
	H = 
\frac{\partial}{\partial x}\left( \left( 0\cdot x + 5\cdot 1\right) \cdot 3\cdot \left( 5\cdot x\right) ^{3 - 1}\right) 
\end{equation}
\begin{equation}
	G = 
\left( 0\cdot x + 5\cdot 1\right) \cdot 3\cdot \left( 5\cdot x\right) ^{3 - 1}
\end{equation}
\begin{equation}
	F = 
\left( 0\cdot x + 5\cdot 1\right) \cdot 3\cdot \left( 5\cdot x\right) ^{3 - 1}
\end{equation}
\begin{equation}
	E = 
1\cdot 5\cdot x^{5 - 1}\cdot \cos {x^{5}}
\end{equation}
\begin{equation}
	D = 
\left( 0\cdot x + 5\cdot 1\right) \cdot 3\cdot \left( 5\cdot x\right) ^{3 - 1}
\end{equation}
\begin{equation}
	C = 
1\cdot 5\cdot x^{5 - 1}\cdot \cos {x^{5}}
\end{equation}
\begin{equation}
	B = 
-1\cdot 1\cdot 5\cdot x^{5 - 1}\cdot \sin {x^{5}}
\end{equation}
\begin{equation}
	A = 
5\cdot 1\cdot \left( 5 - 1\right) \cdot x^{5 - 1 - 1}
\end{equation}
Слава Украине, героям слава:  \begin{equation}
	\frac{\left( \left( \left( 0\cdot 5\cdot x^{5 - 1} + 1\cdot \left( 0\cdot x^{5 - 1} + A\right) \right) \cdot \cos {x^{5}} + 1\cdot 5\cdot x^{5 - 1}\cdot B\right) \cdot \cos {\left( 5\cdot x\right) ^{3}} + C\cdot -1\cdot D\cdot \sin {\left( 5\cdot x\right) ^{3}} - E\cdot -1\cdot F\cdot \sin {\left( 5\cdot x\right) ^{3}} + \sin {x^{5}}\cdot \left( 0\cdot G\cdot \sin {\left( 5\cdot x\right) ^{3}} + -1\cdot \left( \left( H + I\right) \cdot \sin {\left( 5\cdot x\right) ^{3}} + J\cdot \frac{\partial}{\partial x}\left( \sin {\left( 5\cdot x\right) ^{3}}\right) \right) \right) \right) \cdot \left( \cos {\left( 5\cdot x\right) ^{3}}\right) ^{2} - \left( K\cdot \cos {\left( 5\cdot x\right) ^{3}} - \sin {x^{5}}\cdot -1\cdot L\cdot \sin {\left( 5\cdot x\right) ^{3}}\right) \cdot \frac{\partial}{\partial x}\left( \left( \cos {\left( 5\cdot x\right) ^{3}}\right) ^{2}\right) }{\left( \left( \cos {\left( 5\cdot x\right) ^{3}}\right) ^{2}\right) ^{2}}
\end{equation}
Где, 

\begin{equation}
	L = 
\left( 0\cdot x + 5\cdot 1\right) \cdot 3\cdot \left( 5\cdot x\right) ^{3 - 1}
\end{equation}
\begin{equation}
	K = 
1\cdot 5\cdot x^{5 - 1}\cdot \cos {x^{5}}
\end{equation}
\begin{equation}
	J = 
\left( 0\cdot x + 5\cdot 1\right) \cdot 3\cdot \left( 5\cdot x\right) ^{3 - 1}
\end{equation}
\begin{equation}
	I = 
\left( 0\cdot x + 5\cdot 1\right) \cdot \frac{\partial}{\partial x}\left( 3\cdot \left( 5\cdot x\right) ^{3 - 1}\right) 
\end{equation}
\begin{equation}
	H = 
\frac{\partial}{\partial x}\left( 0\cdot x + 5\cdot 1\right) \cdot 3\cdot \left( 5\cdot x\right) ^{3 - 1}
\end{equation}
\begin{equation}
	G = 
\left( 0\cdot x + 5\cdot 1\right) \cdot 3\cdot \left( 5\cdot x\right) ^{3 - 1}
\end{equation}
\begin{equation}
	F = 
\left( 0\cdot x + 5\cdot 1\right) \cdot 3\cdot \left( 5\cdot x\right) ^{3 - 1}
\end{equation}
\begin{equation}
	E = 
1\cdot 5\cdot x^{5 - 1}\cdot \cos {x^{5}}
\end{equation}
\begin{equation}
	D = 
\left( 0\cdot x + 5\cdot 1\right) \cdot 3\cdot \left( 5\cdot x\right) ^{3 - 1}
\end{equation}
\begin{equation}
	C = 
1\cdot 5\cdot x^{5 - 1}\cdot \cos {x^{5}}
\end{equation}
\begin{equation}
	B = 
-1\cdot 1\cdot 5\cdot x^{5 - 1}\cdot \sin {x^{5}}
\end{equation}
\begin{equation}
	A = 
5\cdot 1\cdot \left( 5 - 1\right) \cdot x^{5 - 1 - 1}
\end{equation}
Теорема волшебной палочки гласит, что: По теореме о двух миллиционерах:  \begin{equation}
	\frac{\left( \left( \left( 0\cdot 5\cdot x^{5 - 1} + 1\cdot \left( 0\cdot x^{5 - 1} + A\right) \right) \cdot \cos {x^{5}} + 1\cdot 5\cdot x^{5 - 1}\cdot B\right) \cdot \cos {\left( 5\cdot x\right) ^{3}} + C\cdot -1\cdot D\cdot \sin {\left( 5\cdot x\right) ^{3}} - E\cdot -1\cdot F\cdot \sin {\left( 5\cdot x\right) ^{3}} + \sin {x^{5}}\cdot \left( 0\cdot G\cdot \sin {\left( 5\cdot x\right) ^{3}} + -1\cdot \left( \left( \left( H\right) \cdot 3\cdot \left( 5\cdot x\right) ^{3 - 1} + I\right) \cdot \sin {\left( 5\cdot x\right) ^{3}} + J\cdot \frac{\partial}{\partial x}\left( \sin {\left( 5\cdot x\right) ^{3}}\right) \right) \right) \right) \cdot \left( \cos {\left( 5\cdot x\right) ^{3}}\right) ^{2} - \left( K\cdot \cos {\left( 5\cdot x\right) ^{3}} - \sin {x^{5}}\cdot -1\cdot L\cdot \sin {\left( 5\cdot x\right) ^{3}}\right) \cdot \frac{\partial}{\partial x}\left( \left( \cos {\left( 5\cdot x\right) ^{3}}\right) ^{2}\right) }{\left( \left( \cos {\left( 5\cdot x\right) ^{3}}\right) ^{2}\right) ^{2}}
\end{equation}
Где, 

\begin{equation}
	L = 
\left( 0\cdot x + 5\cdot 1\right) \cdot 3\cdot \left( 5\cdot x\right) ^{3 - 1}
\end{equation}
\begin{equation}
	K = 
1\cdot 5\cdot x^{5 - 1}\cdot \cos {x^{5}}
\end{equation}
\begin{equation}
	J = 
\left( 0\cdot x + 5\cdot 1\right) \cdot 3\cdot \left( 5\cdot x\right) ^{3 - 1}
\end{equation}
\begin{equation}
	I = 
\left( 0\cdot x + 5\cdot 1\right) \cdot \frac{\partial}{\partial x}\left( 3\cdot \left( 5\cdot x\right) ^{3 - 1}\right) 
\end{equation}
\begin{equation}
	H = 
\frac{\partial}{\partial x}\left( 0\cdot x\right)  + \frac{\partial}{\partial x}\left( 5\cdot 1\right) 
\end{equation}
\begin{equation}
	G = 
\left( 0\cdot x + 5\cdot 1\right) \cdot 3\cdot \left( 5\cdot x\right) ^{3 - 1}
\end{equation}
\begin{equation}
	F = 
\left( 0\cdot x + 5\cdot 1\right) \cdot 3\cdot \left( 5\cdot x\right) ^{3 - 1}
\end{equation}
\begin{equation}
	E = 
1\cdot 5\cdot x^{5 - 1}\cdot \cos {x^{5}}
\end{equation}
\begin{equation}
	D = 
\left( 0\cdot x + 5\cdot 1\right) \cdot 3\cdot \left( 5\cdot x\right) ^{3 - 1}
\end{equation}
\begin{equation}
	C = 
1\cdot 5\cdot x^{5 - 1}\cdot \cos {x^{5}}
\end{equation}
\begin{equation}
	B = 
-1\cdot 1\cdot 5\cdot x^{5 - 1}\cdot \sin {x^{5}}
\end{equation}
\begin{equation}
	A = 
5\cdot 1\cdot \left( 5 - 1\right) \cdot x^{5 - 1 - 1}
\end{equation}
Как известно, по теореме Пифагора:  \begin{equation}
	\frac{\left( \left( \left( 0\cdot 5\cdot x^{5 - 1} + 1\cdot \left( 0\cdot x^{5 - 1} + A\right) \right) \cdot \cos {x^{5}} + 1\cdot 5\cdot x^{5 - 1}\cdot B\right) \cdot \cos {\left( 5\cdot x\right) ^{3}} + C\cdot -1\cdot D\cdot \sin {\left( 5\cdot x\right) ^{3}} - E\cdot -1\cdot F\cdot \sin {\left( 5\cdot x\right) ^{3}} + \sin {x^{5}}\cdot \left( 0\cdot G\cdot \sin {\left( 5\cdot x\right) ^{3}} + -1\cdot \left( \left( \left( H\right) \cdot 3\cdot \left( 5\cdot x\right) ^{3 - 1} + I\right) \cdot \sin {\left( 5\cdot x\right) ^{3}} + J\cdot \frac{\partial}{\partial x}\left( \sin {\left( 5\cdot x\right) ^{3}}\right) \right) \right) \right) \cdot \left( \cos {\left( 5\cdot x\right) ^{3}}\right) ^{2} - \left( K\cdot \cos {\left( 5\cdot x\right) ^{3}} - \sin {x^{5}}\cdot -1\cdot L\cdot \sin {\left( 5\cdot x\right) ^{3}}\right) \cdot \frac{\partial}{\partial x}\left( \left( \cos {\left( 5\cdot x\right) ^{3}}\right) ^{2}\right) }{\left( \left( \cos {\left( 5\cdot x\right) ^{3}}\right) ^{2}\right) ^{2}}
\end{equation}
Где, 

\begin{equation}
	L = 
\left( 0\cdot x + 5\cdot 1\right) \cdot 3\cdot \left( 5\cdot x\right) ^{3 - 1}
\end{equation}
\begin{equation}
	K = 
1\cdot 5\cdot x^{5 - 1}\cdot \cos {x^{5}}
\end{equation}
\begin{equation}
	J = 
\left( 0\cdot x + 5\cdot 1\right) \cdot 3\cdot \left( 5\cdot x\right) ^{3 - 1}
\end{equation}
\begin{equation}
	I = 
\left( 0\cdot x + 5\cdot 1\right) \cdot \frac{\partial}{\partial x}\left( 3\cdot \left( 5\cdot x\right) ^{3 - 1}\right) 
\end{equation}
\begin{equation}
	H = 
\frac{\partial}{\partial x}\left( 0\right) \cdot x + 0\cdot \frac{\partial}{\partial x}\left( x\right)  + \frac{\partial}{\partial x}\left( 5\cdot 1\right) 
\end{equation}
\begin{equation}
	G = 
\left( 0\cdot x + 5\cdot 1\right) \cdot 3\cdot \left( 5\cdot x\right) ^{3 - 1}
\end{equation}
\begin{equation}
	F = 
\left( 0\cdot x + 5\cdot 1\right) \cdot 3\cdot \left( 5\cdot x\right) ^{3 - 1}
\end{equation}
\begin{equation}
	E = 
1\cdot 5\cdot x^{5 - 1}\cdot \cos {x^{5}}
\end{equation}
\begin{equation}
	D = 
\left( 0\cdot x + 5\cdot 1\right) \cdot 3\cdot \left( 5\cdot x\right) ^{3 - 1}
\end{equation}
\begin{equation}
	C = 
1\cdot 5\cdot x^{5 - 1}\cdot \cos {x^{5}}
\end{equation}
\begin{equation}
	B = 
-1\cdot 1\cdot 5\cdot x^{5 - 1}\cdot \sin {x^{5}}
\end{equation}
\begin{equation}
	A = 
5\cdot 1\cdot \left( 5 - 1\right) \cdot x^{5 - 1 - 1}
\end{equation}
Согласно теореме о бутерброде с ветчиной:  \begin{equation}
	\frac{\left( \left( \left( 0\cdot 5\cdot x^{5 - 1} + 1\cdot \left( 0\cdot x^{5 - 1} + A\right) \right) \cdot \cos {x^{5}} + 1\cdot 5\cdot x^{5 - 1}\cdot B\right) \cdot \cos {\left( 5\cdot x\right) ^{3}} + C\cdot -1\cdot D\cdot \sin {\left( 5\cdot x\right) ^{3}} - E\cdot -1\cdot F\cdot \sin {\left( 5\cdot x\right) ^{3}} + \sin {x^{5}}\cdot \left( 0\cdot G\cdot \sin {\left( 5\cdot x\right) ^{3}} + -1\cdot \left( \left( \left( H\right) \cdot 3\cdot \left( 5\cdot x\right) ^{3 - 1} + I\right) \cdot \sin {\left( 5\cdot x\right) ^{3}} + J\cdot \frac{\partial}{\partial x}\left( \sin {\left( 5\cdot x\right) ^{3}}\right) \right) \right) \right) \cdot \left( \cos {\left( 5\cdot x\right) ^{3}}\right) ^{2} - \left( K\cdot \cos {\left( 5\cdot x\right) ^{3}} - \sin {x^{5}}\cdot -1\cdot L\cdot \sin {\left( 5\cdot x\right) ^{3}}\right) \cdot \frac{\partial}{\partial x}\left( \left( \cos {\left( 5\cdot x\right) ^{3}}\right) ^{2}\right) }{\left( \left( \cos {\left( 5\cdot x\right) ^{3}}\right) ^{2}\right) ^{2}}
\end{equation}
Где, 

\begin{equation}
	L = 
\left( 0\cdot x + 5\cdot 1\right) \cdot 3\cdot \left( 5\cdot x\right) ^{3 - 1}
\end{equation}
\begin{equation}
	K = 
1\cdot 5\cdot x^{5 - 1}\cdot \cos {x^{5}}
\end{equation}
\begin{equation}
	J = 
\left( 0\cdot x + 5\cdot 1\right) \cdot 3\cdot \left( 5\cdot x\right) ^{3 - 1}
\end{equation}
\begin{equation}
	I = 
\left( 0\cdot x + 5\cdot 1\right) \cdot \frac{\partial}{\partial x}\left( 3\cdot \left( 5\cdot x\right) ^{3 - 1}\right) 
\end{equation}
\begin{equation}
	H = 
0\cdot x + 0\cdot \frac{\partial}{\partial x}\left( x\right)  + \frac{\partial}{\partial x}\left( 5\cdot 1\right) 
\end{equation}
\begin{equation}
	G = 
\left( 0\cdot x + 5\cdot 1\right) \cdot 3\cdot \left( 5\cdot x\right) ^{3 - 1}
\end{equation}
\begin{equation}
	F = 
\left( 0\cdot x + 5\cdot 1\right) \cdot 3\cdot \left( 5\cdot x\right) ^{3 - 1}
\end{equation}
\begin{equation}
	E = 
1\cdot 5\cdot x^{5 - 1}\cdot \cos {x^{5}}
\end{equation}
\begin{equation}
	D = 
\left( 0\cdot x + 5\cdot 1\right) \cdot 3\cdot \left( 5\cdot x\right) ^{3 - 1}
\end{equation}
\begin{equation}
	C = 
1\cdot 5\cdot x^{5 - 1}\cdot \cos {x^{5}}
\end{equation}
\begin{equation}
	B = 
-1\cdot 1\cdot 5\cdot x^{5 - 1}\cdot \sin {x^{5}}
\end{equation}
\begin{equation}
	A = 
5\cdot 1\cdot \left( 5 - 1\right) \cdot x^{5 - 1 - 1}
\end{equation}
Гаусс еще в \RomanNumeralCaps{14} веке посчитал, что:  \begin{equation}
	\frac{\left( \left( \left( 0\cdot 5\cdot x^{5 - 1} + 1\cdot \left( 0\cdot x^{5 - 1} + A\right) \right) \cdot \cos {x^{5}} + 1\cdot 5\cdot x^{5 - 1}\cdot B\right) \cdot \cos {\left( 5\cdot x\right) ^{3}} + C\cdot -1\cdot D\cdot \sin {\left( 5\cdot x\right) ^{3}} - E\cdot -1\cdot F\cdot \sin {\left( 5\cdot x\right) ^{3}} + \sin {x^{5}}\cdot \left( 0\cdot G\cdot \sin {\left( 5\cdot x\right) ^{3}} + -1\cdot \left( \left( \left( H\right) \cdot 3\cdot \left( 5\cdot x\right) ^{3 - 1} + I\right) \cdot \sin {\left( 5\cdot x\right) ^{3}} + J\cdot \frac{\partial}{\partial x}\left( \sin {\left( 5\cdot x\right) ^{3}}\right) \right) \right) \right) \cdot \left( \cos {\left( 5\cdot x\right) ^{3}}\right) ^{2} - \left( K\cdot \cos {\left( 5\cdot x\right) ^{3}} - \sin {x^{5}}\cdot -1\cdot L\cdot \sin {\left( 5\cdot x\right) ^{3}}\right) \cdot \frac{\partial}{\partial x}\left( \left( \cos {\left( 5\cdot x\right) ^{3}}\right) ^{2}\right) }{\left( \left( \cos {\left( 5\cdot x\right) ^{3}}\right) ^{2}\right) ^{2}}
\end{equation}
Где, 

\begin{equation}
	L = 
\left( 0\cdot x + 5\cdot 1\right) \cdot 3\cdot \left( 5\cdot x\right) ^{3 - 1}
\end{equation}
\begin{equation}
	K = 
1\cdot 5\cdot x^{5 - 1}\cdot \cos {x^{5}}
\end{equation}
\begin{equation}
	J = 
\left( 0\cdot x + 5\cdot 1\right) \cdot 3\cdot \left( 5\cdot x\right) ^{3 - 1}
\end{equation}
\begin{equation}
	I = 
\left( 0\cdot x + 5\cdot 1\right) \cdot \frac{\partial}{\partial x}\left( 3\cdot \left( 5\cdot x\right) ^{3 - 1}\right) 
\end{equation}
\begin{equation}
	H = 
0\cdot x + 0\cdot 1 + \frac{\partial}{\partial x}\left( 5\cdot 1\right) 
\end{equation}
\begin{equation}
	G = 
\left( 0\cdot x + 5\cdot 1\right) \cdot 3\cdot \left( 5\cdot x\right) ^{3 - 1}
\end{equation}
\begin{equation}
	F = 
\left( 0\cdot x + 5\cdot 1\right) \cdot 3\cdot \left( 5\cdot x\right) ^{3 - 1}
\end{equation}
\begin{equation}
	E = 
1\cdot 5\cdot x^{5 - 1}\cdot \cos {x^{5}}
\end{equation}
\begin{equation}
	D = 
\left( 0\cdot x + 5\cdot 1\right) \cdot 3\cdot \left( 5\cdot x\right) ^{3 - 1}
\end{equation}
\begin{equation}
	C = 
1\cdot 5\cdot x^{5 - 1}\cdot \cos {x^{5}}
\end{equation}
\begin{equation}
	B = 
-1\cdot 1\cdot 5\cdot x^{5 - 1}\cdot \sin {x^{5}}
\end{equation}
\begin{equation}
	A = 
5\cdot 1\cdot \left( 5 - 1\right) \cdot x^{5 - 1 - 1}
\end{equation}
А Флуктуационно-диссипационная теорема гласит, что:  \begin{equation}
	\frac{\left( \left( \left( 0\cdot 5\cdot x^{5 - 1} + 1\cdot \left( 0\cdot x^{5 - 1} + A\right) \right) \cdot \cos {x^{5}} + 1\cdot 5\cdot x^{5 - 1}\cdot B\right) \cdot \cos {\left( 5\cdot x\right) ^{3}} + C\cdot -1\cdot D\cdot \sin {\left( 5\cdot x\right) ^{3}} - E\cdot -1\cdot F\cdot \sin {\left( 5\cdot x\right) ^{3}} + \sin {x^{5}}\cdot \left( 0\cdot G\cdot \sin {\left( 5\cdot x\right) ^{3}} + -1\cdot \left( \left( \left( H\right) \cdot 3\cdot \left( 5\cdot x\right) ^{3 - 1} + I\right) \cdot \sin {\left( 5\cdot x\right) ^{3}} + J\cdot \frac{\partial}{\partial x}\left( \sin {\left( 5\cdot x\right) ^{3}}\right) \right) \right) \right) \cdot \left( \cos {\left( 5\cdot x\right) ^{3}}\right) ^{2} - \left( K\cdot \cos {\left( 5\cdot x\right) ^{3}} - \sin {x^{5}}\cdot -1\cdot L\cdot \sin {\left( 5\cdot x\right) ^{3}}\right) \cdot \frac{\partial}{\partial x}\left( \left( \cos {\left( 5\cdot x\right) ^{3}}\right) ^{2}\right) }{\left( \left( \cos {\left( 5\cdot x\right) ^{3}}\right) ^{2}\right) ^{2}}
\end{equation}
Где, 

\begin{equation}
	L = 
\left( 0\cdot x + 5\cdot 1\right) \cdot 3\cdot \left( 5\cdot x\right) ^{3 - 1}
\end{equation}
\begin{equation}
	K = 
1\cdot 5\cdot x^{5 - 1}\cdot \cos {x^{5}}
\end{equation}
\begin{equation}
	J = 
\left( 0\cdot x + 5\cdot 1\right) \cdot 3\cdot \left( 5\cdot x\right) ^{3 - 1}
\end{equation}
\begin{equation}
	I = 
\left( 0\cdot x + 5\cdot 1\right) \cdot \frac{\partial}{\partial x}\left( 3\cdot \left( 5\cdot x\right) ^{3 - 1}\right) 
\end{equation}
\begin{equation}
	H = 
0\cdot x + 0\cdot 1 + \frac{\partial}{\partial x}\left( 5\right) \cdot 1 + 5\cdot \frac{\partial}{\partial x}\left( 1\right) 
\end{equation}
\begin{equation}
	G = 
\left( 0\cdot x + 5\cdot 1\right) \cdot 3\cdot \left( 5\cdot x\right) ^{3 - 1}
\end{equation}
\begin{equation}
	F = 
\left( 0\cdot x + 5\cdot 1\right) \cdot 3\cdot \left( 5\cdot x\right) ^{3 - 1}
\end{equation}
\begin{equation}
	E = 
1\cdot 5\cdot x^{5 - 1}\cdot \cos {x^{5}}
\end{equation}
\begin{equation}
	D = 
\left( 0\cdot x + 5\cdot 1\right) \cdot 3\cdot \left( 5\cdot x\right) ^{3 - 1}
\end{equation}
\begin{equation}
	C = 
1\cdot 5\cdot x^{5 - 1}\cdot \cos {x^{5}}
\end{equation}
\begin{equation}
	B = 
-1\cdot 1\cdot 5\cdot x^{5 - 1}\cdot \sin {x^{5}}
\end{equation}
\begin{equation}
	A = 
5\cdot 1\cdot \left( 5 - 1\right) \cdot x^{5 - 1 - 1}
\end{equation}
Как известно, по теореме Пифагора:  \begin{equation}
	\frac{\left( \left( \left( 0\cdot 5\cdot x^{5 - 1} + 1\cdot \left( 0\cdot x^{5 - 1} + A\right) \right) \cdot \cos {x^{5}} + 1\cdot 5\cdot x^{5 - 1}\cdot B\right) \cdot \cos {\left( 5\cdot x\right) ^{3}} + C\cdot -1\cdot D\cdot \sin {\left( 5\cdot x\right) ^{3}} - E\cdot -1\cdot F\cdot \sin {\left( 5\cdot x\right) ^{3}} + \sin {x^{5}}\cdot \left( 0\cdot G\cdot \sin {\left( 5\cdot x\right) ^{3}} + -1\cdot \left( \left( \left( H\right) \cdot 3\cdot \left( 5\cdot x\right) ^{3 - 1} + I\right) \cdot \sin {\left( 5\cdot x\right) ^{3}} + J\cdot \frac{\partial}{\partial x}\left( \sin {\left( 5\cdot x\right) ^{3}}\right) \right) \right) \right) \cdot \left( \cos {\left( 5\cdot x\right) ^{3}}\right) ^{2} - \left( K\cdot \cos {\left( 5\cdot x\right) ^{3}} - \sin {x^{5}}\cdot -1\cdot L\cdot \sin {\left( 5\cdot x\right) ^{3}}\right) \cdot \frac{\partial}{\partial x}\left( \left( \cos {\left( 5\cdot x\right) ^{3}}\right) ^{2}\right) }{\left( \left( \cos {\left( 5\cdot x\right) ^{3}}\right) ^{2}\right) ^{2}}
\end{equation}
Где, 

\begin{equation}
	L = 
\left( 0\cdot x + 5\cdot 1\right) \cdot 3\cdot \left( 5\cdot x\right) ^{3 - 1}
\end{equation}
\begin{equation}
	K = 
1\cdot 5\cdot x^{5 - 1}\cdot \cos {x^{5}}
\end{equation}
\begin{equation}
	J = 
\left( 0\cdot x + 5\cdot 1\right) \cdot 3\cdot \left( 5\cdot x\right) ^{3 - 1}
\end{equation}
\begin{equation}
	I = 
\left( 0\cdot x + 5\cdot 1\right) \cdot \frac{\partial}{\partial x}\left( 3\cdot \left( 5\cdot x\right) ^{3 - 1}\right) 
\end{equation}
\begin{equation}
	H = 
0\cdot x + 0\cdot 1 + 0\cdot 1 + 5\cdot \frac{\partial}{\partial x}\left( 1\right) 
\end{equation}
\begin{equation}
	G = 
\left( 0\cdot x + 5\cdot 1\right) \cdot 3\cdot \left( 5\cdot x\right) ^{3 - 1}
\end{equation}
\begin{equation}
	F = 
\left( 0\cdot x + 5\cdot 1\right) \cdot 3\cdot \left( 5\cdot x\right) ^{3 - 1}
\end{equation}
\begin{equation}
	E = 
1\cdot 5\cdot x^{5 - 1}\cdot \cos {x^{5}}
\end{equation}
\begin{equation}
	D = 
\left( 0\cdot x + 5\cdot 1\right) \cdot 3\cdot \left( 5\cdot x\right) ^{3 - 1}
\end{equation}
\begin{equation}
	C = 
1\cdot 5\cdot x^{5 - 1}\cdot \cos {x^{5}}
\end{equation}
\begin{equation}
	B = 
-1\cdot 1\cdot 5\cdot x^{5 - 1}\cdot \sin {x^{5}}
\end{equation}
\begin{equation}
	A = 
5\cdot 1\cdot \left( 5 - 1\right) \cdot x^{5 - 1 - 1}
\end{equation}
Как известно, по теореме Пифагора:  \begin{equation}
	\frac{\left( \left( \left( 0\cdot 5\cdot x^{5 - 1} + 1\cdot \left( 0\cdot x^{5 - 1} + A\right) \right) \cdot \cos {x^{5}} + 1\cdot 5\cdot x^{5 - 1}\cdot B\right) \cdot \cos {\left( 5\cdot x\right) ^{3}} + C\cdot -1\cdot D\cdot \sin {\left( 5\cdot x\right) ^{3}} - E\cdot -1\cdot F\cdot \sin {\left( 5\cdot x\right) ^{3}} + \sin {x^{5}}\cdot \left( 0\cdot G\cdot \sin {\left( 5\cdot x\right) ^{3}} + -1\cdot \left( \left( \left( H\right) \cdot 3\cdot \left( 5\cdot x\right) ^{3 - 1} + I\right) \cdot \sin {\left( 5\cdot x\right) ^{3}} + J\cdot \frac{\partial}{\partial x}\left( \sin {\left( 5\cdot x\right) ^{3}}\right) \right) \right) \right) \cdot \left( \cos {\left( 5\cdot x\right) ^{3}}\right) ^{2} - \left( K\cdot \cos {\left( 5\cdot x\right) ^{3}} - \sin {x^{5}}\cdot -1\cdot L\cdot \sin {\left( 5\cdot x\right) ^{3}}\right) \cdot \frac{\partial}{\partial x}\left( \left( \cos {\left( 5\cdot x\right) ^{3}}\right) ^{2}\right) }{\left( \left( \cos {\left( 5\cdot x\right) ^{3}}\right) ^{2}\right) ^{2}}
\end{equation}
Где, 

\begin{equation}
	L = 
\left( 0\cdot x + 5\cdot 1\right) \cdot 3\cdot \left( 5\cdot x\right) ^{3 - 1}
\end{equation}
\begin{equation}
	K = 
1\cdot 5\cdot x^{5 - 1}\cdot \cos {x^{5}}
\end{equation}
\begin{equation}
	J = 
\left( 0\cdot x + 5\cdot 1\right) \cdot 3\cdot \left( 5\cdot x\right) ^{3 - 1}
\end{equation}
\begin{equation}
	I = 
\left( 0\cdot x + 5\cdot 1\right) \cdot \frac{\partial}{\partial x}\left( 3\cdot \left( 5\cdot x\right) ^{3 - 1}\right) 
\end{equation}
\begin{equation}
	H = 
0\cdot x + 0\cdot 1 + 0\cdot 1 + 5\cdot 0
\end{equation}
\begin{equation}
	G = 
\left( 0\cdot x + 5\cdot 1\right) \cdot 3\cdot \left( 5\cdot x\right) ^{3 - 1}
\end{equation}
\begin{equation}
	F = 
\left( 0\cdot x + 5\cdot 1\right) \cdot 3\cdot \left( 5\cdot x\right) ^{3 - 1}
\end{equation}
\begin{equation}
	E = 
1\cdot 5\cdot x^{5 - 1}\cdot \cos {x^{5}}
\end{equation}
\begin{equation}
	D = 
\left( 0\cdot x + 5\cdot 1\right) \cdot 3\cdot \left( 5\cdot x\right) ^{3 - 1}
\end{equation}
\begin{equation}
	C = 
1\cdot 5\cdot x^{5 - 1}\cdot \cos {x^{5}}
\end{equation}
\begin{equation}
	B = 
-1\cdot 1\cdot 5\cdot x^{5 - 1}\cdot \sin {x^{5}}
\end{equation}
\begin{equation}
	A = 
5\cdot 1\cdot \left( 5 - 1\right) \cdot x^{5 - 1 - 1}
\end{equation}
Как известно, по теореме Пифагора:  \begin{equation}
	\frac{\left( \left( \left( 0\cdot 5\cdot x^{5 - 1} + 1\cdot \left( 0\cdot x^{5 - 1} + A\right) \right) \cdot \cos {x^{5}} + 1\cdot 5\cdot x^{5 - 1}\cdot B\right) \cdot \cos {\left( 5\cdot x\right) ^{3}} + C\cdot -1\cdot D\cdot \sin {\left( 5\cdot x\right) ^{3}} - E\cdot -1\cdot F\cdot \sin {\left( 5\cdot x\right) ^{3}} + \sin {x^{5}}\cdot \left( 0\cdot G\cdot \sin {\left( 5\cdot x\right) ^{3}} + -1\cdot \left( \left( \left( H\right) \cdot 3\cdot \left( 5\cdot x\right) ^{3 - 1} + \left( 0\cdot x + 5\cdot 1\right) \cdot \left( I + J\right) \right) \cdot \sin {\left( 5\cdot x\right) ^{3}} + K\cdot \frac{\partial}{\partial x}\left( \sin {\left( 5\cdot x\right) ^{3}}\right) \right) \right) \right) \cdot \left( \cos {\left( 5\cdot x\right) ^{3}}\right) ^{2} - \left( L\cdot \cos {\left( 5\cdot x\right) ^{3}} - \sin {x^{5}}\cdot -1\cdot M\cdot \sin {\left( 5\cdot x\right) ^{3}}\right) \cdot \frac{\partial}{\partial x}\left( \left( \cos {\left( 5\cdot x\right) ^{3}}\right) ^{2}\right) }{\left( \left( \cos {\left( 5\cdot x\right) ^{3}}\right) ^{2}\right) ^{2}}
\end{equation}
Где, 

\begin{equation}
	M = 
\left( 0\cdot x + 5\cdot 1\right) \cdot 3\cdot \left( 5\cdot x\right) ^{3 - 1}
\end{equation}
\begin{equation}
	L = 
1\cdot 5\cdot x^{5 - 1}\cdot \cos {x^{5}}
\end{equation}
\begin{equation}
	K = 
\left( 0\cdot x + 5\cdot 1\right) \cdot 3\cdot \left( 5\cdot x\right) ^{3 - 1}
\end{equation}
\begin{equation}
	J = 
3\cdot \frac{\partial}{\partial x}\left( \left( 5\cdot x\right) ^{3 - 1}\right) 
\end{equation}
\begin{equation}
	I = 
\frac{\partial}{\partial x}\left( 3\right) \cdot \left( 5\cdot x\right) ^{3 - 1}
\end{equation}
\begin{equation}
	H = 
0\cdot x + 0\cdot 1 + 0\cdot 1 + 5\cdot 0
\end{equation}
\begin{equation}
	G = 
\left( 0\cdot x + 5\cdot 1\right) \cdot 3\cdot \left( 5\cdot x\right) ^{3 - 1}
\end{equation}
\begin{equation}
	F = 
\left( 0\cdot x + 5\cdot 1\right) \cdot 3\cdot \left( 5\cdot x\right) ^{3 - 1}
\end{equation}
\begin{equation}
	E = 
1\cdot 5\cdot x^{5 - 1}\cdot \cos {x^{5}}
\end{equation}
\begin{equation}
	D = 
\left( 0\cdot x + 5\cdot 1\right) \cdot 3\cdot \left( 5\cdot x\right) ^{3 - 1}
\end{equation}
\begin{equation}
	C = 
1\cdot 5\cdot x^{5 - 1}\cdot \cos {x^{5}}
\end{equation}
\begin{equation}
	B = 
-1\cdot 1\cdot 5\cdot x^{5 - 1}\cdot \sin {x^{5}}
\end{equation}
\begin{equation}
	A = 
5\cdot 1\cdot \left( 5 - 1\right) \cdot x^{5 - 1 - 1}
\end{equation}
Очевидно, что по критерию Сильвестра:  \begin{equation}
	\frac{\left( \left( \left( 0\cdot 5\cdot x^{5 - 1} + 1\cdot \left( 0\cdot x^{5 - 1} + A\right) \right) \cdot \cos {x^{5}} + 1\cdot 5\cdot x^{5 - 1}\cdot B\right) \cdot \cos {\left( 5\cdot x\right) ^{3}} + C\cdot -1\cdot D\cdot \sin {\left( 5\cdot x\right) ^{3}} - E\cdot -1\cdot F\cdot \sin {\left( 5\cdot x\right) ^{3}} + \sin {x^{5}}\cdot \left( 0\cdot G\cdot \sin {\left( 5\cdot x\right) ^{3}} + -1\cdot \left( \left( \left( H\right) \cdot 3\cdot \left( 5\cdot x\right) ^{3 - 1} + \left( 0\cdot x + 5\cdot 1\right) \cdot \left( 0\cdot \left( 5\cdot x\right) ^{3 - 1} + I\right) \right) \cdot \sin {\left( 5\cdot x\right) ^{3}} + J\cdot \frac{\partial}{\partial x}\left( \sin {\left( 5\cdot x\right) ^{3}}\right) \right) \right) \right) \cdot \left( \cos {\left( 5\cdot x\right) ^{3}}\right) ^{2} - \left( K\cdot \cos {\left( 5\cdot x\right) ^{3}} - \sin {x^{5}}\cdot -1\cdot L\cdot \sin {\left( 5\cdot x\right) ^{3}}\right) \cdot \frac{\partial}{\partial x}\left( \left( \cos {\left( 5\cdot x\right) ^{3}}\right) ^{2}\right) }{\left( \left( \cos {\left( 5\cdot x\right) ^{3}}\right) ^{2}\right) ^{2}}
\end{equation}
Где, 

\begin{equation}
	L = 
\left( 0\cdot x + 5\cdot 1\right) \cdot 3\cdot \left( 5\cdot x\right) ^{3 - 1}
\end{equation}
\begin{equation}
	K = 
1\cdot 5\cdot x^{5 - 1}\cdot \cos {x^{5}}
\end{equation}
\begin{equation}
	J = 
\left( 0\cdot x + 5\cdot 1\right) \cdot 3\cdot \left( 5\cdot x\right) ^{3 - 1}
\end{equation}
\begin{equation}
	I = 
3\cdot \frac{\partial}{\partial x}\left( \left( 5\cdot x\right) ^{3 - 1}\right) 
\end{equation}
\begin{equation}
	H = 
0\cdot x + 0\cdot 1 + 0\cdot 1 + 5\cdot 0
\end{equation}
\begin{equation}
	G = 
\left( 0\cdot x + 5\cdot 1\right) \cdot 3\cdot \left( 5\cdot x\right) ^{3 - 1}
\end{equation}
\begin{equation}
	F = 
\left( 0\cdot x + 5\cdot 1\right) \cdot 3\cdot \left( 5\cdot x\right) ^{3 - 1}
\end{equation}
\begin{equation}
	E = 
1\cdot 5\cdot x^{5 - 1}\cdot \cos {x^{5}}
\end{equation}
\begin{equation}
	D = 
\left( 0\cdot x + 5\cdot 1\right) \cdot 3\cdot \left( 5\cdot x\right) ^{3 - 1}
\end{equation}
\begin{equation}
	C = 
1\cdot 5\cdot x^{5 - 1}\cdot \cos {x^{5}}
\end{equation}
\begin{equation}
	B = 
-1\cdot 1\cdot 5\cdot x^{5 - 1}\cdot \sin {x^{5}}
\end{equation}
\begin{equation}
	A = 
5\cdot 1\cdot \left( 5 - 1\right) \cdot x^{5 - 1 - 1}
\end{equation}
Очевидно, что по критерию Сильвестра:  \begin{equation}
	\frac{\left( \left( \left( 0\cdot 5\cdot x^{5 - 1} + 1\cdot \left( 0\cdot x^{5 - 1} + A\right) \right) \cdot \cos {x^{5}} + 1\cdot 5\cdot x^{5 - 1}\cdot B\right) \cdot \cos {\left( 5\cdot x\right) ^{3}} + C\cdot -1\cdot D\cdot \sin {\left( 5\cdot x\right) ^{3}} - E\cdot -1\cdot F\cdot \sin {\left( 5\cdot x\right) ^{3}} + \sin {x^{5}}\cdot \left( 0\cdot G\cdot \sin {\left( 5\cdot x\right) ^{3}} + -1\cdot \left( \left( \left( H\right) \cdot 3\cdot \left( 5\cdot x\right) ^{3 - 1} + \left( 0\cdot x + 5\cdot 1\right) \cdot \left( 0\cdot \left( 5\cdot x\right) ^{3 - 1} + 3\cdot I\right) \right) \cdot \sin {\left( 5\cdot x\right) ^{3}} + J\cdot \frac{\partial}{\partial x}\left( \sin {\left( 5\cdot x\right) ^{3}}\right) \right) \right) \right) \cdot \left( \cos {\left( 5\cdot x\right) ^{3}}\right) ^{2} - \left( K\cdot \cos {\left( 5\cdot x\right) ^{3}} - \sin {x^{5}}\cdot -1\cdot L\cdot \sin {\left( 5\cdot x\right) ^{3}}\right) \cdot \frac{\partial}{\partial x}\left( \left( \cos {\left( 5\cdot x\right) ^{3}}\right) ^{2}\right) }{\left( \left( \cos {\left( 5\cdot x\right) ^{3}}\right) ^{2}\right) ^{2}}
\end{equation}
Где, 

\begin{equation}
	L = 
\left( 0\cdot x + 5\cdot 1\right) \cdot 3\cdot \left( 5\cdot x\right) ^{3 - 1}
\end{equation}
\begin{equation}
	K = 
1\cdot 5\cdot x^{5 - 1}\cdot \cos {x^{5}}
\end{equation}
\begin{equation}
	J = 
\left( 0\cdot x + 5\cdot 1\right) \cdot 3\cdot \left( 5\cdot x\right) ^{3 - 1}
\end{equation}
\begin{equation}
	I = 
\frac{\partial}{\partial x}\left( 5\cdot x\right) \cdot \left( 3 - 1\right) \cdot \left( 5\cdot x\right) ^{3 - 1 - 1}
\end{equation}
\begin{equation}
	H = 
0\cdot x + 0\cdot 1 + 0\cdot 1 + 5\cdot 0
\end{equation}
\begin{equation}
	G = 
\left( 0\cdot x + 5\cdot 1\right) \cdot 3\cdot \left( 5\cdot x\right) ^{3 - 1}
\end{equation}
\begin{equation}
	F = 
\left( 0\cdot x + 5\cdot 1\right) \cdot 3\cdot \left( 5\cdot x\right) ^{3 - 1}
\end{equation}
\begin{equation}
	E = 
1\cdot 5\cdot x^{5 - 1}\cdot \cos {x^{5}}
\end{equation}
\begin{equation}
	D = 
\left( 0\cdot x + 5\cdot 1\right) \cdot 3\cdot \left( 5\cdot x\right) ^{3 - 1}
\end{equation}
\begin{equation}
	C = 
1\cdot 5\cdot x^{5 - 1}\cdot \cos {x^{5}}
\end{equation}
\begin{equation}
	B = 
-1\cdot 1\cdot 5\cdot x^{5 - 1}\cdot \sin {x^{5}}
\end{equation}
\begin{equation}
	A = 
5\cdot 1\cdot \left( 5 - 1\right) \cdot x^{5 - 1 - 1}
\end{equation}
Теорема волшебной палочки гласит, что: По теореме о двух миллиционерах:  \begin{equation}
	\frac{\left( \left( \left( 0\cdot 5\cdot x^{5 - 1} + 1\cdot \left( 0\cdot x^{5 - 1} + A\right) \right) \cdot \cos {x^{5}} + 1\cdot 5\cdot x^{5 - 1}\cdot B\right) \cdot \cos {\left( 5\cdot x\right) ^{3}} + C\cdot -1\cdot D\cdot \sin {\left( 5\cdot x\right) ^{3}} - E\cdot -1\cdot F\cdot \sin {\left( 5\cdot x\right) ^{3}} + \sin {x^{5}}\cdot \left( 0\cdot G\cdot \sin {\left( 5\cdot x\right) ^{3}} + -1\cdot \left( \left( \left( H\right) \cdot 3\cdot \left( 5\cdot x\right) ^{3 - 1} + \left( 0\cdot x + 5\cdot 1\right) \cdot \left( 0\cdot \left( 5\cdot x\right) ^{3 - 1} + 3\cdot \left( I\right) \cdot J\right) \right) \cdot \sin {\left( 5\cdot x\right) ^{3}} + K\cdot \frac{\partial}{\partial x}\left( \sin {\left( 5\cdot x\right) ^{3}}\right) \right) \right) \right) \cdot \left( \cos {\left( 5\cdot x\right) ^{3}}\right) ^{2} - \left( L\cdot \cos {\left( 5\cdot x\right) ^{3}} - \sin {x^{5}}\cdot -1\cdot M\cdot \sin {\left( 5\cdot x\right) ^{3}}\right) \cdot \frac{\partial}{\partial x}\left( \left( \cos {\left( 5\cdot x\right) ^{3}}\right) ^{2}\right) }{\left( \left( \cos {\left( 5\cdot x\right) ^{3}}\right) ^{2}\right) ^{2}}
\end{equation}
Где, 

\begin{equation}
	M = 
\left( 0\cdot x + 5\cdot 1\right) \cdot 3\cdot \left( 5\cdot x\right) ^{3 - 1}
\end{equation}
\begin{equation}
	L = 
1\cdot 5\cdot x^{5 - 1}\cdot \cos {x^{5}}
\end{equation}
\begin{equation}
	K = 
\left( 0\cdot x + 5\cdot 1\right) \cdot 3\cdot \left( 5\cdot x\right) ^{3 - 1}
\end{equation}
\begin{equation}
	J = 
\left( 3 - 1\right) \cdot \left( 5\cdot x\right) ^{3 - 1 - 1}
\end{equation}
\begin{equation}
	I = 
\frac{\partial}{\partial x}\left( 5\right) \cdot x + 5\cdot \frac{\partial}{\partial x}\left( x\right) 
\end{equation}
\begin{equation}
	H = 
0\cdot x + 0\cdot 1 + 0\cdot 1 + 5\cdot 0
\end{equation}
\begin{equation}
	G = 
\left( 0\cdot x + 5\cdot 1\right) \cdot 3\cdot \left( 5\cdot x\right) ^{3 - 1}
\end{equation}
\begin{equation}
	F = 
\left( 0\cdot x + 5\cdot 1\right) \cdot 3\cdot \left( 5\cdot x\right) ^{3 - 1}
\end{equation}
\begin{equation}
	E = 
1\cdot 5\cdot x^{5 - 1}\cdot \cos {x^{5}}
\end{equation}
\begin{equation}
	D = 
\left( 0\cdot x + 5\cdot 1\right) \cdot 3\cdot \left( 5\cdot x\right) ^{3 - 1}
\end{equation}
\begin{equation}
	C = 
1\cdot 5\cdot x^{5 - 1}\cdot \cos {x^{5}}
\end{equation}
\begin{equation}
	B = 
-1\cdot 1\cdot 5\cdot x^{5 - 1}\cdot \sin {x^{5}}
\end{equation}
\begin{equation}
	A = 
5\cdot 1\cdot \left( 5 - 1\right) \cdot x^{5 - 1 - 1}
\end{equation}
Очевидно, что по критерию Сильвестра:  \begin{equation}
	\frac{\left( \left( \left( 0\cdot 5\cdot x^{5 - 1} + 1\cdot \left( 0\cdot x^{5 - 1} + A\right) \right) \cdot \cos {x^{5}} + 1\cdot 5\cdot x^{5 - 1}\cdot B\right) \cdot \cos {\left( 5\cdot x\right) ^{3}} + C\cdot -1\cdot D\cdot \sin {\left( 5\cdot x\right) ^{3}} - E\cdot -1\cdot F\cdot \sin {\left( 5\cdot x\right) ^{3}} + \sin {x^{5}}\cdot \left( 0\cdot G\cdot \sin {\left( 5\cdot x\right) ^{3}} + -1\cdot \left( \left( \left( H\right) \cdot 3\cdot \left( 5\cdot x\right) ^{3 - 1} + \left( 0\cdot x + 5\cdot 1\right) \cdot \left( 0\cdot \left( 5\cdot x\right) ^{3 - 1} + 3\cdot \left( 0\cdot x + 5\cdot \frac{\partial}{\partial x}\left( x\right) \right) \cdot I\right) \right) \cdot \sin {\left( 5\cdot x\right) ^{3}} + J\cdot \frac{\partial}{\partial x}\left( \sin {\left( 5\cdot x\right) ^{3}}\right) \right) \right) \right) \cdot \left( \cos {\left( 5\cdot x\right) ^{3}}\right) ^{2} - \left( K\cdot \cos {\left( 5\cdot x\right) ^{3}} - \sin {x^{5}}\cdot -1\cdot L\cdot \sin {\left( 5\cdot x\right) ^{3}}\right) \cdot \frac{\partial}{\partial x}\left( \left( \cos {\left( 5\cdot x\right) ^{3}}\right) ^{2}\right) }{\left( \left( \cos {\left( 5\cdot x\right) ^{3}}\right) ^{2}\right) ^{2}}
\end{equation}
Где, 

\begin{equation}
	L = 
\left( 0\cdot x + 5\cdot 1\right) \cdot 3\cdot \left( 5\cdot x\right) ^{3 - 1}
\end{equation}
\begin{equation}
	K = 
1\cdot 5\cdot x^{5 - 1}\cdot \cos {x^{5}}
\end{equation}
\begin{equation}
	J = 
\left( 0\cdot x + 5\cdot 1\right) \cdot 3\cdot \left( 5\cdot x\right) ^{3 - 1}
\end{equation}
\begin{equation}
	I = 
\left( 3 - 1\right) \cdot \left( 5\cdot x\right) ^{3 - 1 - 1}
\end{equation}
\begin{equation}
	H = 
0\cdot x + 0\cdot 1 + 0\cdot 1 + 5\cdot 0
\end{equation}
\begin{equation}
	G = 
\left( 0\cdot x + 5\cdot 1\right) \cdot 3\cdot \left( 5\cdot x\right) ^{3 - 1}
\end{equation}
\begin{equation}
	F = 
\left( 0\cdot x + 5\cdot 1\right) \cdot 3\cdot \left( 5\cdot x\right) ^{3 - 1}
\end{equation}
\begin{equation}
	E = 
1\cdot 5\cdot x^{5 - 1}\cdot \cos {x^{5}}
\end{equation}
\begin{equation}
	D = 
\left( 0\cdot x + 5\cdot 1\right) \cdot 3\cdot \left( 5\cdot x\right) ^{3 - 1}
\end{equation}
\begin{equation}
	C = 
1\cdot 5\cdot x^{5 - 1}\cdot \cos {x^{5}}
\end{equation}
\begin{equation}
	B = 
-1\cdot 1\cdot 5\cdot x^{5 - 1}\cdot \sin {x^{5}}
\end{equation}
\begin{equation}
	A = 
5\cdot 1\cdot \left( 5 - 1\right) \cdot x^{5 - 1 - 1}
\end{equation}
По теореме Лиувилля о сохранении фазового объёма:  \begin{equation}
	\frac{\left( \left( \left( 0\cdot 5\cdot x^{5 - 1} + 1\cdot \left( 0\cdot x^{5 - 1} + A\right) \right) \cdot \cos {x^{5}} + 1\cdot 5\cdot x^{5 - 1}\cdot B\right) \cdot \cos {\left( 5\cdot x\right) ^{3}} + C\cdot -1\cdot D\cdot \sin {\left( 5\cdot x\right) ^{3}} - E\cdot -1\cdot F\cdot \sin {\left( 5\cdot x\right) ^{3}} + \sin {x^{5}}\cdot \left( 0\cdot G\cdot \sin {\left( 5\cdot x\right) ^{3}} + -1\cdot \left( \left( \left( H\right) \cdot 3\cdot \left( 5\cdot x\right) ^{3 - 1} + \left( 0\cdot x + 5\cdot 1\right) \cdot \left( 0\cdot \left( 5\cdot x\right) ^{3 - 1} + 3\cdot \left( 0\cdot x + 5\cdot 1\right) \cdot I\right) \right) \cdot \sin {\left( 5\cdot x\right) ^{3}} + J\cdot \frac{\partial}{\partial x}\left( \sin {\left( 5\cdot x\right) ^{3}}\right) \right) \right) \right) \cdot \left( \cos {\left( 5\cdot x\right) ^{3}}\right) ^{2} - \left( K\cdot \cos {\left( 5\cdot x\right) ^{3}} - \sin {x^{5}}\cdot -1\cdot L\cdot \sin {\left( 5\cdot x\right) ^{3}}\right) \cdot \frac{\partial}{\partial x}\left( \left( \cos {\left( 5\cdot x\right) ^{3}}\right) ^{2}\right) }{\left( \left( \cos {\left( 5\cdot x\right) ^{3}}\right) ^{2}\right) ^{2}}
\end{equation}
Где, 

\begin{equation}
	L = 
\left( 0\cdot x + 5\cdot 1\right) \cdot 3\cdot \left( 5\cdot x\right) ^{3 - 1}
\end{equation}
\begin{equation}
	K = 
1\cdot 5\cdot x^{5 - 1}\cdot \cos {x^{5}}
\end{equation}
\begin{equation}
	J = 
\left( 0\cdot x + 5\cdot 1\right) \cdot 3\cdot \left( 5\cdot x\right) ^{3 - 1}
\end{equation}
\begin{equation}
	I = 
\left( 3 - 1\right) \cdot \left( 5\cdot x\right) ^{3 - 1 - 1}
\end{equation}
\begin{equation}
	H = 
0\cdot x + 0\cdot 1 + 0\cdot 1 + 5\cdot 0
\end{equation}
\begin{equation}
	G = 
\left( 0\cdot x + 5\cdot 1\right) \cdot 3\cdot \left( 5\cdot x\right) ^{3 - 1}
\end{equation}
\begin{equation}
	F = 
\left( 0\cdot x + 5\cdot 1\right) \cdot 3\cdot \left( 5\cdot x\right) ^{3 - 1}
\end{equation}
\begin{equation}
	E = 
1\cdot 5\cdot x^{5 - 1}\cdot \cos {x^{5}}
\end{equation}
\begin{equation}
	D = 
\left( 0\cdot x + 5\cdot 1\right) \cdot 3\cdot \left( 5\cdot x\right) ^{3 - 1}
\end{equation}
\begin{equation}
	C = 
1\cdot 5\cdot x^{5 - 1}\cdot \cos {x^{5}}
\end{equation}
\begin{equation}
	B = 
-1\cdot 1\cdot 5\cdot x^{5 - 1}\cdot \sin {x^{5}}
\end{equation}
\begin{equation}
	A = 
5\cdot 1\cdot \left( 5 - 1\right) \cdot x^{5 - 1 - 1}
\end{equation}
Как известно, по теореме Пифагора:  \begin{equation}
	\frac{\left( \left( \left( 0\cdot 5\cdot x^{5 - 1} + 1\cdot \left( 0\cdot x^{5 - 1} + A\right) \right) \cdot \cos {x^{5}} + 1\cdot 5\cdot x^{5 - 1}\cdot B\right) \cdot \cos {\left( 5\cdot x\right) ^{3}} + C\cdot -1\cdot D\cdot \sin {\left( 5\cdot x\right) ^{3}} - E\cdot -1\cdot F\cdot \sin {\left( 5\cdot x\right) ^{3}} + \sin {x^{5}}\cdot \left( 0\cdot G\cdot \sin {\left( 5\cdot x\right) ^{3}} + -1\cdot \left( \left( \left( H\right) \cdot 3\cdot \left( 5\cdot x\right) ^{3 - 1} + \left( 0\cdot x + 5\cdot 1\right) \cdot \left( 0\cdot \left( 5\cdot x\right) ^{3 - 1} + 3\cdot \left( 0\cdot x + 5\cdot 1\right) \cdot I\right) \right) \cdot \sin {\left( 5\cdot x\right) ^{3}} + J\cdot K\right) \right) \right) \cdot \left( \cos {\left( 5\cdot x\right) ^{3}}\right) ^{2} - \left( L\cdot \cos {\left( 5\cdot x\right) ^{3}} - \sin {x^{5}}\cdot -1\cdot M\cdot \sin {\left( 5\cdot x\right) ^{3}}\right) \cdot \frac{\partial}{\partial x}\left( \left( \cos {\left( 5\cdot x\right) ^{3}}\right) ^{2}\right) }{\left( \left( \cos {\left( 5\cdot x\right) ^{3}}\right) ^{2}\right) ^{2}}
\end{equation}
Где, 

\begin{equation}
	M = 
\left( 0\cdot x + 5\cdot 1\right) \cdot 3\cdot \left( 5\cdot x\right) ^{3 - 1}
\end{equation}
\begin{equation}
	L = 
1\cdot 5\cdot x^{5 - 1}\cdot \cos {x^{5}}
\end{equation}
\begin{equation}
	K = 
\frac{\partial}{\partial x}\left( \left( 5\cdot x\right) ^{3}\right) \cdot \cos {\left( 5\cdot x\right) ^{3}}
\end{equation}
\begin{equation}
	J = 
\left( 0\cdot x + 5\cdot 1\right) \cdot 3\cdot \left( 5\cdot x\right) ^{3 - 1}
\end{equation}
\begin{equation}
	I = 
\left( 3 - 1\right) \cdot \left( 5\cdot x\right) ^{3 - 1 - 1}
\end{equation}
\begin{equation}
	H = 
0\cdot x + 0\cdot 1 + 0\cdot 1 + 5\cdot 0
\end{equation}
\begin{equation}
	G = 
\left( 0\cdot x + 5\cdot 1\right) \cdot 3\cdot \left( 5\cdot x\right) ^{3 - 1}
\end{equation}
\begin{equation}
	F = 
\left( 0\cdot x + 5\cdot 1\right) \cdot 3\cdot \left( 5\cdot x\right) ^{3 - 1}
\end{equation}
\begin{equation}
	E = 
1\cdot 5\cdot x^{5 - 1}\cdot \cos {x^{5}}
\end{equation}
\begin{equation}
	D = 
\left( 0\cdot x + 5\cdot 1\right) \cdot 3\cdot \left( 5\cdot x\right) ^{3 - 1}
\end{equation}
\begin{equation}
	C = 
1\cdot 5\cdot x^{5 - 1}\cdot \cos {x^{5}}
\end{equation}
\begin{equation}
	B = 
-1\cdot 1\cdot 5\cdot x^{5 - 1}\cdot \sin {x^{5}}
\end{equation}
\begin{equation}
	A = 
5\cdot 1\cdot \left( 5 - 1\right) \cdot x^{5 - 1 - 1}
\end{equation}
По теореме о причёсывании ежа:  \begin{equation}
	\frac{\left( \left( \left( 0\cdot 5\cdot x^{5 - 1} + 1\cdot \left( 0\cdot x^{5 - 1} + A\right) \right) \cdot \cos {x^{5}} + 1\cdot 5\cdot x^{5 - 1}\cdot B\right) \cdot \cos {\left( 5\cdot x\right) ^{3}} + C\cdot -1\cdot D\cdot \sin {\left( 5\cdot x\right) ^{3}} - E\cdot -1\cdot F\cdot \sin {\left( 5\cdot x\right) ^{3}} + \sin {x^{5}}\cdot \left( 0\cdot G\cdot \sin {\left( 5\cdot x\right) ^{3}} + -1\cdot \left( \left( \left( H\right) \cdot 3\cdot \left( 5\cdot x\right) ^{3 - 1} + \left( 0\cdot x + 5\cdot 1\right) \cdot \left( 0\cdot \left( 5\cdot x\right) ^{3 - 1} + 3\cdot \left( 0\cdot x + 5\cdot 1\right) \cdot I\right) \right) \cdot \sin {\left( 5\cdot x\right) ^{3}} + J\cdot K\cdot \cos {\left( 5\cdot x\right) ^{3}}\right) \right) \right) \cdot \left( \cos {\left( 5\cdot x\right) ^{3}}\right) ^{2} - \left( L\cdot \cos {\left( 5\cdot x\right) ^{3}} - \sin {x^{5}}\cdot -1\cdot M\cdot \sin {\left( 5\cdot x\right) ^{3}}\right) \cdot \frac{\partial}{\partial x}\left( \left( \cos {\left( 5\cdot x\right) ^{3}}\right) ^{2}\right) }{\left( \left( \cos {\left( 5\cdot x\right) ^{3}}\right) ^{2}\right) ^{2}}
\end{equation}
Где, 

\begin{equation}
	M = 
\left( 0\cdot x + 5\cdot 1\right) \cdot 3\cdot \left( 5\cdot x\right) ^{3 - 1}
\end{equation}
\begin{equation}
	L = 
1\cdot 5\cdot x^{5 - 1}\cdot \cos {x^{5}}
\end{equation}
\begin{equation}
	K = 
\frac{\partial}{\partial x}\left( 5\cdot x\right) \cdot 3\cdot \left( 5\cdot x\right) ^{3 - 1}
\end{equation}
\begin{equation}
	J = 
\left( 0\cdot x + 5\cdot 1\right) \cdot 3\cdot \left( 5\cdot x\right) ^{3 - 1}
\end{equation}
\begin{equation}
	I = 
\left( 3 - 1\right) \cdot \left( 5\cdot x\right) ^{3 - 1 - 1}
\end{equation}
\begin{equation}
	H = 
0\cdot x + 0\cdot 1 + 0\cdot 1 + 5\cdot 0
\end{equation}
\begin{equation}
	G = 
\left( 0\cdot x + 5\cdot 1\right) \cdot 3\cdot \left( 5\cdot x\right) ^{3 - 1}
\end{equation}
\begin{equation}
	F = 
\left( 0\cdot x + 5\cdot 1\right) \cdot 3\cdot \left( 5\cdot x\right) ^{3 - 1}
\end{equation}
\begin{equation}
	E = 
1\cdot 5\cdot x^{5 - 1}\cdot \cos {x^{5}}
\end{equation}
\begin{equation}
	D = 
\left( 0\cdot x + 5\cdot 1\right) \cdot 3\cdot \left( 5\cdot x\right) ^{3 - 1}
\end{equation}
\begin{equation}
	C = 
1\cdot 5\cdot x^{5 - 1}\cdot \cos {x^{5}}
\end{equation}
\begin{equation}
	B = 
-1\cdot 1\cdot 5\cdot x^{5 - 1}\cdot \sin {x^{5}}
\end{equation}
\begin{equation}
	A = 
5\cdot 1\cdot \left( 5 - 1\right) \cdot x^{5 - 1 - 1}
\end{equation}
А по теореме Лиувилля об интеграле уравнения Гамильтона — Якоби:  \begin{equation}
	\frac{\left( \left( \left( 0\cdot 5\cdot x^{5 - 1} + 1\cdot \left( 0\cdot x^{5 - 1} + A\right) \right) \cdot \cos {x^{5}} + 1\cdot 5\cdot x^{5 - 1}\cdot B\right) \cdot \cos {\left( 5\cdot x\right) ^{3}} + C\cdot -1\cdot D\cdot \sin {\left( 5\cdot x\right) ^{3}} - E\cdot -1\cdot F\cdot \sin {\left( 5\cdot x\right) ^{3}} + \sin {x^{5}}\cdot \left( 0\cdot G\cdot \sin {\left( 5\cdot x\right) ^{3}} + -1\cdot \left( \left( \left( H\right) \cdot 3\cdot \left( 5\cdot x\right) ^{3 - 1} + \left( 0\cdot x + 5\cdot 1\right) \cdot \left( 0\cdot \left( 5\cdot x\right) ^{3 - 1} + 3\cdot \left( 0\cdot x + 5\cdot 1\right) \cdot I\right) \right) \cdot \sin {\left( 5\cdot x\right) ^{3}} + J\cdot \left( K\right) \cdot 3\cdot \left( 5\cdot x\right) ^{3 - 1}\cdot \cos {\left( 5\cdot x\right) ^{3}}\right) \right) \right) \cdot \left( \cos {\left( 5\cdot x\right) ^{3}}\right) ^{2} - \left( L\cdot \cos {\left( 5\cdot x\right) ^{3}} - \sin {x^{5}}\cdot -1\cdot M\cdot \sin {\left( 5\cdot x\right) ^{3}}\right) \cdot \frac{\partial}{\partial x}\left( \left( \cos {\left( 5\cdot x\right) ^{3}}\right) ^{2}\right) }{\left( \left( \cos {\left( 5\cdot x\right) ^{3}}\right) ^{2}\right) ^{2}}
\end{equation}
Где, 

\begin{equation}
	M = 
\left( 0\cdot x + 5\cdot 1\right) \cdot 3\cdot \left( 5\cdot x\right) ^{3 - 1}
\end{equation}
\begin{equation}
	L = 
1\cdot 5\cdot x^{5 - 1}\cdot \cos {x^{5}}
\end{equation}
\begin{equation}
	K = 
\frac{\partial}{\partial x}\left( 5\right) \cdot x + 5\cdot \frac{\partial}{\partial x}\left( x\right) 
\end{equation}
\begin{equation}
	J = 
\left( 0\cdot x + 5\cdot 1\right) \cdot 3\cdot \left( 5\cdot x\right) ^{3 - 1}
\end{equation}
\begin{equation}
	I = 
\left( 3 - 1\right) \cdot \left( 5\cdot x\right) ^{3 - 1 - 1}
\end{equation}
\begin{equation}
	H = 
0\cdot x + 0\cdot 1 + 0\cdot 1 + 5\cdot 0
\end{equation}
\begin{equation}
	G = 
\left( 0\cdot x + 5\cdot 1\right) \cdot 3\cdot \left( 5\cdot x\right) ^{3 - 1}
\end{equation}
\begin{equation}
	F = 
\left( 0\cdot x + 5\cdot 1\right) \cdot 3\cdot \left( 5\cdot x\right) ^{3 - 1}
\end{equation}
\begin{equation}
	E = 
1\cdot 5\cdot x^{5 - 1}\cdot \cos {x^{5}}
\end{equation}
\begin{equation}
	D = 
\left( 0\cdot x + 5\cdot 1\right) \cdot 3\cdot \left( 5\cdot x\right) ^{3 - 1}
\end{equation}
\begin{equation}
	C = 
1\cdot 5\cdot x^{5 - 1}\cdot \cos {x^{5}}
\end{equation}
\begin{equation}
	B = 
-1\cdot 1\cdot 5\cdot x^{5 - 1}\cdot \sin {x^{5}}
\end{equation}
\begin{equation}
	A = 
5\cdot 1\cdot \left( 5 - 1\right) \cdot x^{5 - 1 - 1}
\end{equation}
Очевидно, что по критерию Сильвестра:  \begin{equation}
	\frac{\left( \left( \left( 0\cdot 5\cdot x^{5 - 1} + 1\cdot \left( 0\cdot x^{5 - 1} + A\right) \right) \cdot \cos {x^{5}} + 1\cdot 5\cdot x^{5 - 1}\cdot B\right) \cdot \cos {\left( 5\cdot x\right) ^{3}} + C\cdot -1\cdot D\cdot \sin {\left( 5\cdot x\right) ^{3}} - E\cdot -1\cdot F\cdot \sin {\left( 5\cdot x\right) ^{3}} + \sin {x^{5}}\cdot \left( 0\cdot G\cdot \sin {\left( 5\cdot x\right) ^{3}} + -1\cdot \left( \left( \left( H\right) \cdot 3\cdot \left( 5\cdot x\right) ^{3 - 1} + \left( 0\cdot x + 5\cdot 1\right) \cdot \left( 0\cdot \left( 5\cdot x\right) ^{3 - 1} + 3\cdot \left( 0\cdot x + 5\cdot 1\right) \cdot I\right) \right) \cdot \sin {\left( 5\cdot x\right) ^{3}} + J\cdot K\cdot \cos {\left( 5\cdot x\right) ^{3}}\right) \right) \right) \cdot \left( \cos {\left( 5\cdot x\right) ^{3}}\right) ^{2} - \left( L\cdot \cos {\left( 5\cdot x\right) ^{3}} - \sin {x^{5}}\cdot -1\cdot M\cdot \sin {\left( 5\cdot x\right) ^{3}}\right) \cdot \frac{\partial}{\partial x}\left( \left( \cos {\left( 5\cdot x\right) ^{3}}\right) ^{2}\right) }{\left( \left( \cos {\left( 5\cdot x\right) ^{3}}\right) ^{2}\right) ^{2}}
\end{equation}
Где, 

\begin{equation}
	M = 
\left( 0\cdot x + 5\cdot 1\right) \cdot 3\cdot \left( 5\cdot x\right) ^{3 - 1}
\end{equation}
\begin{equation}
	L = 
1\cdot 5\cdot x^{5 - 1}\cdot \cos {x^{5}}
\end{equation}
\begin{equation}
	K = 
\left( 0\cdot x + 5\cdot \frac{\partial}{\partial x}\left( x\right) \right) \cdot 3\cdot \left( 5\cdot x\right) ^{3 - 1}
\end{equation}
\begin{equation}
	J = 
\left( 0\cdot x + 5\cdot 1\right) \cdot 3\cdot \left( 5\cdot x\right) ^{3 - 1}
\end{equation}
\begin{equation}
	I = 
\left( 3 - 1\right) \cdot \left( 5\cdot x\right) ^{3 - 1 - 1}
\end{equation}
\begin{equation}
	H = 
0\cdot x + 0\cdot 1 + 0\cdot 1 + 5\cdot 0
\end{equation}
\begin{equation}
	G = 
\left( 0\cdot x + 5\cdot 1\right) \cdot 3\cdot \left( 5\cdot x\right) ^{3 - 1}
\end{equation}
\begin{equation}
	F = 
\left( 0\cdot x + 5\cdot 1\right) \cdot 3\cdot \left( 5\cdot x\right) ^{3 - 1}
\end{equation}
\begin{equation}
	E = 
1\cdot 5\cdot x^{5 - 1}\cdot \cos {x^{5}}
\end{equation}
\begin{equation}
	D = 
\left( 0\cdot x + 5\cdot 1\right) \cdot 3\cdot \left( 5\cdot x\right) ^{3 - 1}
\end{equation}
\begin{equation}
	C = 
1\cdot 5\cdot x^{5 - 1}\cdot \cos {x^{5}}
\end{equation}
\begin{equation}
	B = 
-1\cdot 1\cdot 5\cdot x^{5 - 1}\cdot \sin {x^{5}}
\end{equation}
\begin{equation}
	A = 
5\cdot 1\cdot \left( 5 - 1\right) \cdot x^{5 - 1 - 1}
\end{equation}
Гаусс еще в \RomanNumeralCaps{14} веке посчитал, что:  \begin{equation}
	\frac{\left( \left( \left( 0\cdot 5\cdot x^{5 - 1} + 1\cdot \left( 0\cdot x^{5 - 1} + A\right) \right) \cdot \cos {x^{5}} + 1\cdot 5\cdot x^{5 - 1}\cdot B\right) \cdot \cos {\left( 5\cdot x\right) ^{3}} + C\cdot -1\cdot D\cdot \sin {\left( 5\cdot x\right) ^{3}} - E\cdot -1\cdot F\cdot \sin {\left( 5\cdot x\right) ^{3}} + \sin {x^{5}}\cdot \left( 0\cdot G\cdot \sin {\left( 5\cdot x\right) ^{3}} + -1\cdot \left( \left( \left( H\right) \cdot 3\cdot \left( 5\cdot x\right) ^{3 - 1} + \left( 0\cdot x + 5\cdot 1\right) \cdot \left( 0\cdot \left( 5\cdot x\right) ^{3 - 1} + 3\cdot \left( 0\cdot x + 5\cdot 1\right) \cdot I\right) \right) \cdot \sin {\left( 5\cdot x\right) ^{3}} + J\cdot K\cdot \cos {\left( 5\cdot x\right) ^{3}}\right) \right) \right) \cdot \left( \cos {\left( 5\cdot x\right) ^{3}}\right) ^{2} - \left( L\cdot \cos {\left( 5\cdot x\right) ^{3}} - \sin {x^{5}}\cdot -1\cdot M\cdot \sin {\left( 5\cdot x\right) ^{3}}\right) \cdot \frac{\partial}{\partial x}\left( \left( \cos {\left( 5\cdot x\right) ^{3}}\right) ^{2}\right) }{\left( \left( \cos {\left( 5\cdot x\right) ^{3}}\right) ^{2}\right) ^{2}}
\end{equation}
Где, 

\begin{equation}
	M = 
\left( 0\cdot x + 5\cdot 1\right) \cdot 3\cdot \left( 5\cdot x\right) ^{3 - 1}
\end{equation}
\begin{equation}
	L = 
1\cdot 5\cdot x^{5 - 1}\cdot \cos {x^{5}}
\end{equation}
\begin{equation}
	K = 
\left( 0\cdot x + 5\cdot 1\right) \cdot 3\cdot \left( 5\cdot x\right) ^{3 - 1}
\end{equation}
\begin{equation}
	J = 
\left( 0\cdot x + 5\cdot 1\right) \cdot 3\cdot \left( 5\cdot x\right) ^{3 - 1}
\end{equation}
\begin{equation}
	I = 
\left( 3 - 1\right) \cdot \left( 5\cdot x\right) ^{3 - 1 - 1}
\end{equation}
\begin{equation}
	H = 
0\cdot x + 0\cdot 1 + 0\cdot 1 + 5\cdot 0
\end{equation}
\begin{equation}
	G = 
\left( 0\cdot x + 5\cdot 1\right) \cdot 3\cdot \left( 5\cdot x\right) ^{3 - 1}
\end{equation}
\begin{equation}
	F = 
\left( 0\cdot x + 5\cdot 1\right) \cdot 3\cdot \left( 5\cdot x\right) ^{3 - 1}
\end{equation}
\begin{equation}
	E = 
1\cdot 5\cdot x^{5 - 1}\cdot \cos {x^{5}}
\end{equation}
\begin{equation}
	D = 
\left( 0\cdot x + 5\cdot 1\right) \cdot 3\cdot \left( 5\cdot x\right) ^{3 - 1}
\end{equation}
\begin{equation}
	C = 
1\cdot 5\cdot x^{5 - 1}\cdot \cos {x^{5}}
\end{equation}
\begin{equation}
	B = 
-1\cdot 1\cdot 5\cdot x^{5 - 1}\cdot \sin {x^{5}}
\end{equation}
\begin{equation}
	A = 
5\cdot 1\cdot \left( 5 - 1\right) \cdot x^{5 - 1 - 1}
\end{equation}
Согласно теореме о бутерброде с ветчиной:  \begin{equation}
	\frac{\left( \left( \left( 0\cdot 5\cdot x^{5 - 1} + 1\cdot \left( 0\cdot x^{5 - 1} + A\right) \right) \cdot \cos {x^{5}} + 1\cdot 5\cdot x^{5 - 1}\cdot B\right) \cdot \cos {\left( 5\cdot x\right) ^{3}} + C\cdot -1\cdot D\cdot \sin {\left( 5\cdot x\right) ^{3}} - E\cdot -1\cdot F\cdot \sin {\left( 5\cdot x\right) ^{3}} + \sin {x^{5}}\cdot \left( 0\cdot G\cdot \sin {\left( 5\cdot x\right) ^{3}} + -1\cdot \left( \left( \left( H\right) \cdot 3\cdot \left( 5\cdot x\right) ^{3 - 1} + \left( 0\cdot x + 5\cdot 1\right) \cdot \left( 0\cdot \left( 5\cdot x\right) ^{3 - 1} + 3\cdot \left( 0\cdot x + 5\cdot 1\right) \cdot I\right) \right) \cdot \sin {\left( 5\cdot x\right) ^{3}} + J\cdot K\cdot \cos {\left( 5\cdot x\right) ^{3}}\right) \right) \right) \cdot \left( \cos {\left( 5\cdot x\right) ^{3}}\right) ^{2} - \left( L\cdot \cos {\left( 5\cdot x\right) ^{3}} - \sin {x^{5}}\cdot -1\cdot M\cdot \sin {\left( 5\cdot x\right) ^{3}}\right) \cdot \frac{\partial}{\partial x}\left( \cos {\left( 5\cdot x\right) ^{3}}\right) \cdot N}{\left( \left( \cos {\left( 5\cdot x\right) ^{3}}\right) ^{2}\right) ^{2}}
\end{equation}
Где, 

\begin{equation}
	N = 
2\cdot \left( \cos {\left( 5\cdot x\right) ^{3}}\right) ^{2 - 1}
\end{equation}
\begin{equation}
	M = 
\left( 0\cdot x + 5\cdot 1\right) \cdot 3\cdot \left( 5\cdot x\right) ^{3 - 1}
\end{equation}
\begin{equation}
	L = 
1\cdot 5\cdot x^{5 - 1}\cdot \cos {x^{5}}
\end{equation}
\begin{equation}
	K = 
\left( 0\cdot x + 5\cdot 1\right) \cdot 3\cdot \left( 5\cdot x\right) ^{3 - 1}
\end{equation}
\begin{equation}
	J = 
\left( 0\cdot x + 5\cdot 1\right) \cdot 3\cdot \left( 5\cdot x\right) ^{3 - 1}
\end{equation}
\begin{equation}
	I = 
\left( 3 - 1\right) \cdot \left( 5\cdot x\right) ^{3 - 1 - 1}
\end{equation}
\begin{equation}
	H = 
0\cdot x + 0\cdot 1 + 0\cdot 1 + 5\cdot 0
\end{equation}
\begin{equation}
	G = 
\left( 0\cdot x + 5\cdot 1\right) \cdot 3\cdot \left( 5\cdot x\right) ^{3 - 1}
\end{equation}
\begin{equation}
	F = 
\left( 0\cdot x + 5\cdot 1\right) \cdot 3\cdot \left( 5\cdot x\right) ^{3 - 1}
\end{equation}
\begin{equation}
	E = 
1\cdot 5\cdot x^{5 - 1}\cdot \cos {x^{5}}
\end{equation}
\begin{equation}
	D = 
\left( 0\cdot x + 5\cdot 1\right) \cdot 3\cdot \left( 5\cdot x\right) ^{3 - 1}
\end{equation}
\begin{equation}
	C = 
1\cdot 5\cdot x^{5 - 1}\cdot \cos {x^{5}}
\end{equation}
\begin{equation}
	B = 
-1\cdot 1\cdot 5\cdot x^{5 - 1}\cdot \sin {x^{5}}
\end{equation}
\begin{equation}
	A = 
5\cdot 1\cdot \left( 5 - 1\right) \cdot x^{5 - 1 - 1}
\end{equation}
Как известно, по теореме Пифагора:  \begin{equation}
	\frac{\left( \left( \left( 0\cdot 5\cdot x^{5 - 1} + 1\cdot \left( 0\cdot x^{5 - 1} + A\right) \right) \cdot \cos {x^{5}} + 1\cdot 5\cdot x^{5 - 1}\cdot B\right) \cdot \cos {\left( 5\cdot x\right) ^{3}} + C\cdot -1\cdot D\cdot \sin {\left( 5\cdot x\right) ^{3}} - E\cdot -1\cdot F\cdot \sin {\left( 5\cdot x\right) ^{3}} + \sin {x^{5}}\cdot \left( 0\cdot G\cdot \sin {\left( 5\cdot x\right) ^{3}} + -1\cdot \left( \left( \left( H\right) \cdot 3\cdot \left( 5\cdot x\right) ^{3 - 1} + \left( 0\cdot x + 5\cdot 1\right) \cdot \left( 0\cdot \left( 5\cdot x\right) ^{3 - 1} + 3\cdot \left( 0\cdot x + 5\cdot 1\right) \cdot I\right) \right) \cdot \sin {\left( 5\cdot x\right) ^{3}} + J\cdot K\cdot \cos {\left( 5\cdot x\right) ^{3}}\right) \right) \right) \cdot \left( \cos {\left( 5\cdot x\right) ^{3}}\right) ^{2} - \left( L\cdot \cos {\left( 5\cdot x\right) ^{3}} - \sin {x^{5}}\cdot -1\cdot M\cdot \sin {\left( 5\cdot x\right) ^{3}}\right) \cdot N\cdot O}{\left( \left( \cos {\left( 5\cdot x\right) ^{3}}\right) ^{2}\right) ^{2}}
\end{equation}
Где, 

\begin{equation}
	O = 
2\cdot \left( \cos {\left( 5\cdot x\right) ^{3}}\right) ^{2 - 1}
\end{equation}
\begin{equation}
	N = 
-1\cdot \frac{\partial}{\partial x}\left( \left( 5\cdot x\right) ^{3}\right) \cdot \sin {\left( 5\cdot x\right) ^{3}}
\end{equation}
\begin{equation}
	M = 
\left( 0\cdot x + 5\cdot 1\right) \cdot 3\cdot \left( 5\cdot x\right) ^{3 - 1}
\end{equation}
\begin{equation}
	L = 
1\cdot 5\cdot x^{5 - 1}\cdot \cos {x^{5}}
\end{equation}
\begin{equation}
	K = 
\left( 0\cdot x + 5\cdot 1\right) \cdot 3\cdot \left( 5\cdot x\right) ^{3 - 1}
\end{equation}
\begin{equation}
	J = 
\left( 0\cdot x + 5\cdot 1\right) \cdot 3\cdot \left( 5\cdot x\right) ^{3 - 1}
\end{equation}
\begin{equation}
	I = 
\left( 3 - 1\right) \cdot \left( 5\cdot x\right) ^{3 - 1 - 1}
\end{equation}
\begin{equation}
	H = 
0\cdot x + 0\cdot 1 + 0\cdot 1 + 5\cdot 0
\end{equation}
\begin{equation}
	G = 
\left( 0\cdot x + 5\cdot 1\right) \cdot 3\cdot \left( 5\cdot x\right) ^{3 - 1}
\end{equation}
\begin{equation}
	F = 
\left( 0\cdot x + 5\cdot 1\right) \cdot 3\cdot \left( 5\cdot x\right) ^{3 - 1}
\end{equation}
\begin{equation}
	E = 
1\cdot 5\cdot x^{5 - 1}\cdot \cos {x^{5}}
\end{equation}
\begin{equation}
	D = 
\left( 0\cdot x + 5\cdot 1\right) \cdot 3\cdot \left( 5\cdot x\right) ^{3 - 1}
\end{equation}
\begin{equation}
	C = 
1\cdot 5\cdot x^{5 - 1}\cdot \cos {x^{5}}
\end{equation}
\begin{equation}
	B = 
-1\cdot 1\cdot 5\cdot x^{5 - 1}\cdot \sin {x^{5}}
\end{equation}
\begin{equation}
	A = 
5\cdot 1\cdot \left( 5 - 1\right) \cdot x^{5 - 1 - 1}
\end{equation}
Как известно, по теореме Пифагора:  \begin{equation}
	\frac{\left( \left( \left( 0\cdot 5\cdot x^{5 - 1} + 1\cdot \left( 0\cdot x^{5 - 1} + A\right) \right) \cdot \cos {x^{5}} + 1\cdot 5\cdot x^{5 - 1}\cdot B\right) \cdot \cos {\left( 5\cdot x\right) ^{3}} + C\cdot -1\cdot D\cdot \sin {\left( 5\cdot x\right) ^{3}} - E\cdot -1\cdot F\cdot \sin {\left( 5\cdot x\right) ^{3}} + \sin {x^{5}}\cdot \left( 0\cdot G\cdot \sin {\left( 5\cdot x\right) ^{3}} + -1\cdot \left( \left( \left( H\right) \cdot 3\cdot \left( 5\cdot x\right) ^{3 - 1} + \left( 0\cdot x + 5\cdot 1\right) \cdot \left( 0\cdot \left( 5\cdot x\right) ^{3 - 1} + 3\cdot \left( 0\cdot x + 5\cdot 1\right) \cdot I\right) \right) \cdot \sin {\left( 5\cdot x\right) ^{3}} + J\cdot K\cdot \cos {\left( 5\cdot x\right) ^{3}}\right) \right) \right) \cdot \left( \cos {\left( 5\cdot x\right) ^{3}}\right) ^{2} - \left( L\cdot \cos {\left( 5\cdot x\right) ^{3}} - \sin {x^{5}}\cdot -1\cdot M\cdot \sin {\left( 5\cdot x\right) ^{3}}\right) \cdot -1\cdot N\cdot \sin {\left( 5\cdot x\right) ^{3}}\cdot O}{\left( \left( \cos {\left( 5\cdot x\right) ^{3}}\right) ^{2}\right) ^{2}}
\end{equation}
Где, 

\begin{equation}
	O = 
2\cdot \left( \cos {\left( 5\cdot x\right) ^{3}}\right) ^{2 - 1}
\end{equation}
\begin{equation}
	N = 
\frac{\partial}{\partial x}\left( 5\cdot x\right) \cdot 3\cdot \left( 5\cdot x\right) ^{3 - 1}
\end{equation}
\begin{equation}
	M = 
\left( 0\cdot x + 5\cdot 1\right) \cdot 3\cdot \left( 5\cdot x\right) ^{3 - 1}
\end{equation}
\begin{equation}
	L = 
1\cdot 5\cdot x^{5 - 1}\cdot \cos {x^{5}}
\end{equation}
\begin{equation}
	K = 
\left( 0\cdot x + 5\cdot 1\right) \cdot 3\cdot \left( 5\cdot x\right) ^{3 - 1}
\end{equation}
\begin{equation}
	J = 
\left( 0\cdot x + 5\cdot 1\right) \cdot 3\cdot \left( 5\cdot x\right) ^{3 - 1}
\end{equation}
\begin{equation}
	I = 
\left( 3 - 1\right) \cdot \left( 5\cdot x\right) ^{3 - 1 - 1}
\end{equation}
\begin{equation}
	H = 
0\cdot x + 0\cdot 1 + 0\cdot 1 + 5\cdot 0
\end{equation}
\begin{equation}
	G = 
\left( 0\cdot x + 5\cdot 1\right) \cdot 3\cdot \left( 5\cdot x\right) ^{3 - 1}
\end{equation}
\begin{equation}
	F = 
\left( 0\cdot x + 5\cdot 1\right) \cdot 3\cdot \left( 5\cdot x\right) ^{3 - 1}
\end{equation}
\begin{equation}
	E = 
1\cdot 5\cdot x^{5 - 1}\cdot \cos {x^{5}}
\end{equation}
\begin{equation}
	D = 
\left( 0\cdot x + 5\cdot 1\right) \cdot 3\cdot \left( 5\cdot x\right) ^{3 - 1}
\end{equation}
\begin{equation}
	C = 
1\cdot 5\cdot x^{5 - 1}\cdot \cos {x^{5}}
\end{equation}
\begin{equation}
	B = 
-1\cdot 1\cdot 5\cdot x^{5 - 1}\cdot \sin {x^{5}}
\end{equation}
\begin{equation}
	A = 
5\cdot 1\cdot \left( 5 - 1\right) \cdot x^{5 - 1 - 1}
\end{equation}
Очевидно, что по критерию Сильвестра:  \begin{equation}
	\frac{\left( \left( \left( 0\cdot 5\cdot x^{5 - 1} + 1\cdot \left( 0\cdot x^{5 - 1} + A\right) \right) \cdot \cos {x^{5}} + 1\cdot 5\cdot x^{5 - 1}\cdot B\right) \cdot \cos {\left( 5\cdot x\right) ^{3}} + C\cdot -1\cdot D\cdot \sin {\left( 5\cdot x\right) ^{3}} - E\cdot -1\cdot F\cdot \sin {\left( 5\cdot x\right) ^{3}} + \sin {x^{5}}\cdot \left( 0\cdot G\cdot \sin {\left( 5\cdot x\right) ^{3}} + -1\cdot \left( \left( \left( H\right) \cdot 3\cdot \left( 5\cdot x\right) ^{3 - 1} + \left( 0\cdot x + 5\cdot 1\right) \cdot \left( 0\cdot \left( 5\cdot x\right) ^{3 - 1} + 3\cdot \left( 0\cdot x + 5\cdot 1\right) \cdot I\right) \right) \cdot \sin {\left( 5\cdot x\right) ^{3}} + J\cdot K\cdot \cos {\left( 5\cdot x\right) ^{3}}\right) \right) \right) \cdot \left( \cos {\left( 5\cdot x\right) ^{3}}\right) ^{2} - \left( L\cdot \cos {\left( 5\cdot x\right) ^{3}} - \sin {x^{5}}\cdot -1\cdot M\cdot \sin {\left( 5\cdot x\right) ^{3}}\right) \cdot -1\cdot \left( N\right) \cdot 3\cdot \left( 5\cdot x\right) ^{3 - 1}\cdot \sin {\left( 5\cdot x\right) ^{3}}\cdot O}{\left( \left( \cos {\left( 5\cdot x\right) ^{3}}\right) ^{2}\right) ^{2}}
\end{equation}
Где, 

\begin{equation}
	O = 
2\cdot \left( \cos {\left( 5\cdot x\right) ^{3}}\right) ^{2 - 1}
\end{equation}
\begin{equation}
	N = 
\frac{\partial}{\partial x}\left( 5\right) \cdot x + 5\cdot \frac{\partial}{\partial x}\left( x\right) 
\end{equation}
\begin{equation}
	M = 
\left( 0\cdot x + 5\cdot 1\right) \cdot 3\cdot \left( 5\cdot x\right) ^{3 - 1}
\end{equation}
\begin{equation}
	L = 
1\cdot 5\cdot x^{5 - 1}\cdot \cos {x^{5}}
\end{equation}
\begin{equation}
	K = 
\left( 0\cdot x + 5\cdot 1\right) \cdot 3\cdot \left( 5\cdot x\right) ^{3 - 1}
\end{equation}
\begin{equation}
	J = 
\left( 0\cdot x + 5\cdot 1\right) \cdot 3\cdot \left( 5\cdot x\right) ^{3 - 1}
\end{equation}
\begin{equation}
	I = 
\left( 3 - 1\right) \cdot \left( 5\cdot x\right) ^{3 - 1 - 1}
\end{equation}
\begin{equation}
	H = 
0\cdot x + 0\cdot 1 + 0\cdot 1 + 5\cdot 0
\end{equation}
\begin{equation}
	G = 
\left( 0\cdot x + 5\cdot 1\right) \cdot 3\cdot \left( 5\cdot x\right) ^{3 - 1}
\end{equation}
\begin{equation}
	F = 
\left( 0\cdot x + 5\cdot 1\right) \cdot 3\cdot \left( 5\cdot x\right) ^{3 - 1}
\end{equation}
\begin{equation}
	E = 
1\cdot 5\cdot x^{5 - 1}\cdot \cos {x^{5}}
\end{equation}
\begin{equation}
	D = 
\left( 0\cdot x + 5\cdot 1\right) \cdot 3\cdot \left( 5\cdot x\right) ^{3 - 1}
\end{equation}
\begin{equation}
	C = 
1\cdot 5\cdot x^{5 - 1}\cdot \cos {x^{5}}
\end{equation}
\begin{equation}
	B = 
-1\cdot 1\cdot 5\cdot x^{5 - 1}\cdot \sin {x^{5}}
\end{equation}
\begin{equation}
	A = 
5\cdot 1\cdot \left( 5 - 1\right) \cdot x^{5 - 1 - 1}
\end{equation}
А по теореме Лиувилля об интеграле уравнения Гамильтона — Якоби:  \begin{equation}
	\frac{\left( \left( \left( 0\cdot 5\cdot x^{5 - 1} + 1\cdot \left( 0\cdot x^{5 - 1} + A\right) \right) \cdot \cos {x^{5}} + 1\cdot 5\cdot x^{5 - 1}\cdot B\right) \cdot \cos {\left( 5\cdot x\right) ^{3}} + C\cdot -1\cdot D\cdot \sin {\left( 5\cdot x\right) ^{3}} - E\cdot -1\cdot F\cdot \sin {\left( 5\cdot x\right) ^{3}} + \sin {x^{5}}\cdot \left( 0\cdot G\cdot \sin {\left( 5\cdot x\right) ^{3}} + -1\cdot \left( \left( \left( H\right) \cdot 3\cdot \left( 5\cdot x\right) ^{3 - 1} + \left( 0\cdot x + 5\cdot 1\right) \cdot \left( 0\cdot \left( 5\cdot x\right) ^{3 - 1} + 3\cdot \left( 0\cdot x + 5\cdot 1\right) \cdot I\right) \right) \cdot \sin {\left( 5\cdot x\right) ^{3}} + J\cdot K\cdot \cos {\left( 5\cdot x\right) ^{3}}\right) \right) \right) \cdot \left( \cos {\left( 5\cdot x\right) ^{3}}\right) ^{2} - \left( L\cdot \cos {\left( 5\cdot x\right) ^{3}} - \sin {x^{5}}\cdot -1\cdot M\cdot \sin {\left( 5\cdot x\right) ^{3}}\right) \cdot -1\cdot N\cdot \sin {\left( 5\cdot x\right) ^{3}}\cdot O}{\left( \left( \cos {\left( 5\cdot x\right) ^{3}}\right) ^{2}\right) ^{2}}
\end{equation}
Где, 

\begin{equation}
	O = 
2\cdot \left( \cos {\left( 5\cdot x\right) ^{3}}\right) ^{2 - 1}
\end{equation}
\begin{equation}
	N = 
\left( 0\cdot x + 5\cdot \frac{\partial}{\partial x}\left( x\right) \right) \cdot 3\cdot \left( 5\cdot x\right) ^{3 - 1}
\end{equation}
\begin{equation}
	M = 
\left( 0\cdot x + 5\cdot 1\right) \cdot 3\cdot \left( 5\cdot x\right) ^{3 - 1}
\end{equation}
\begin{equation}
	L = 
1\cdot 5\cdot x^{5 - 1}\cdot \cos {x^{5}}
\end{equation}
\begin{equation}
	K = 
\left( 0\cdot x + 5\cdot 1\right) \cdot 3\cdot \left( 5\cdot x\right) ^{3 - 1}
\end{equation}
\begin{equation}
	J = 
\left( 0\cdot x + 5\cdot 1\right) \cdot 3\cdot \left( 5\cdot x\right) ^{3 - 1}
\end{equation}
\begin{equation}
	I = 
\left( 3 - 1\right) \cdot \left( 5\cdot x\right) ^{3 - 1 - 1}
\end{equation}
\begin{equation}
	H = 
0\cdot x + 0\cdot 1 + 0\cdot 1 + 5\cdot 0
\end{equation}
\begin{equation}
	G = 
\left( 0\cdot x + 5\cdot 1\right) \cdot 3\cdot \left( 5\cdot x\right) ^{3 - 1}
\end{equation}
\begin{equation}
	F = 
\left( 0\cdot x + 5\cdot 1\right) \cdot 3\cdot \left( 5\cdot x\right) ^{3 - 1}
\end{equation}
\begin{equation}
	E = 
1\cdot 5\cdot x^{5 - 1}\cdot \cos {x^{5}}
\end{equation}
\begin{equation}
	D = 
\left( 0\cdot x + 5\cdot 1\right) \cdot 3\cdot \left( 5\cdot x\right) ^{3 - 1}
\end{equation}
\begin{equation}
	C = 
1\cdot 5\cdot x^{5 - 1}\cdot \cos {x^{5}}
\end{equation}
\begin{equation}
	B = 
-1\cdot 1\cdot 5\cdot x^{5 - 1}\cdot \sin {x^{5}}
\end{equation}
\begin{equation}
	A = 
5\cdot 1\cdot \left( 5 - 1\right) \cdot x^{5 - 1 - 1}
\end{equation}
А Флуктуационно-диссипационная теорема гласит, что:  \begin{equation}
	\frac{\left( \left( \left( 0\cdot 5\cdot x^{5 - 1} + 1\cdot \left( 0\cdot x^{5 - 1} + A\right) \right) \cdot \cos {x^{5}} + 1\cdot 5\cdot x^{5 - 1}\cdot B\right) \cdot \cos {\left( 5\cdot x\right) ^{3}} + C\cdot -1\cdot D\cdot \sin {\left( 5\cdot x\right) ^{3}} - E\cdot -1\cdot F\cdot \sin {\left( 5\cdot x\right) ^{3}} + \sin {x^{5}}\cdot \left( 0\cdot G\cdot \sin {\left( 5\cdot x\right) ^{3}} + -1\cdot \left( \left( \left( H\right) \cdot 3\cdot \left( 5\cdot x\right) ^{3 - 1} + \left( 0\cdot x + 5\cdot 1\right) \cdot \left( 0\cdot \left( 5\cdot x\right) ^{3 - 1} + 3\cdot \left( 0\cdot x + 5\cdot 1\right) \cdot I\right) \right) \cdot \sin {\left( 5\cdot x\right) ^{3}} + J\cdot K\cdot \cos {\left( 5\cdot x\right) ^{3}}\right) \right) \right) \cdot \left( \cos {\left( 5\cdot x\right) ^{3}}\right) ^{2} - \left( L\cdot \cos {\left( 5\cdot x\right) ^{3}} - \sin {x^{5}}\cdot -1\cdot M\cdot \sin {\left( 5\cdot x\right) ^{3}}\right) \cdot -1\cdot N\cdot \sin {\left( 5\cdot x\right) ^{3}}\cdot O}{\left( \left( \cos {\left( 5\cdot x\right) ^{3}}\right) ^{2}\right) ^{2}}
\end{equation}
Где, 

\begin{equation}
	O = 
2\cdot \left( \cos {\left( 5\cdot x\right) ^{3}}\right) ^{2 - 1}
\end{equation}
\begin{equation}
	N = 
\left( 0\cdot x + 5\cdot 1\right) \cdot 3\cdot \left( 5\cdot x\right) ^{3 - 1}
\end{equation}
\begin{equation}
	M = 
\left( 0\cdot x + 5\cdot 1\right) \cdot 3\cdot \left( 5\cdot x\right) ^{3 - 1}
\end{equation}
\begin{equation}
	L = 
1\cdot 5\cdot x^{5 - 1}\cdot \cos {x^{5}}
\end{equation}
\begin{equation}
	K = 
\left( 0\cdot x + 5\cdot 1\right) \cdot 3\cdot \left( 5\cdot x\right) ^{3 - 1}
\end{equation}
\begin{equation}
	J = 
\left( 0\cdot x + 5\cdot 1\right) \cdot 3\cdot \left( 5\cdot x\right) ^{3 - 1}
\end{equation}
\begin{equation}
	I = 
\left( 3 - 1\right) \cdot \left( 5\cdot x\right) ^{3 - 1 - 1}
\end{equation}
\begin{equation}
	H = 
0\cdot x + 0\cdot 1 + 0\cdot 1 + 5\cdot 0
\end{equation}
\begin{equation}
	G = 
\left( 0\cdot x + 5\cdot 1\right) \cdot 3\cdot \left( 5\cdot x\right) ^{3 - 1}
\end{equation}
\begin{equation}
	F = 
\left( 0\cdot x + 5\cdot 1\right) \cdot 3\cdot \left( 5\cdot x\right) ^{3 - 1}
\end{equation}
\begin{equation}
	E = 
1\cdot 5\cdot x^{5 - 1}\cdot \cos {x^{5}}
\end{equation}
\begin{equation}
	D = 
\left( 0\cdot x + 5\cdot 1\right) \cdot 3\cdot \left( 5\cdot x\right) ^{3 - 1}
\end{equation}
\begin{equation}
	C = 
1\cdot 5\cdot x^{5 - 1}\cdot \cos {x^{5}}
\end{equation}
\begin{equation}
	B = 
-1\cdot 1\cdot 5\cdot x^{5 - 1}\cdot \sin {x^{5}}
\end{equation}
\begin{equation}
	A = 
5\cdot 1\cdot \left( 5 - 1\right) \cdot x^{5 - 1 - 1}
\end{equation}
Предлагаем читателю убедиться в том, что конечный ответ: 
\begin{equation}
	\frac{\left( \left( A + B\right) \cdot \cos {\left( 5\cdot x\right) ^{3}} + 5\cdot x^{4}\cdot \cos {x^{5}}\cdot C - 5\cdot x^{4}\cdot \cos {x^{5}}\cdot D + \sin {x^{5}}\cdot -1\cdot \left( E + 5\cdot 3\cdot \left( 5\cdot x\right) ^{2}\cdot F\right) \right) \cdot \left( \cos {\left( 5\cdot x\right) ^{3}}\right) ^{2} - \left( G - \sin {x^{5}}\cdot H\right) \cdot I\cdot 2\cdot \cos {\left( 5\cdot x\right) ^{3}}}{\left( \left( \cos {\left( 5\cdot x\right) ^{3}}\right) ^{2}\right) ^{2}}
\end{equation}
Где, 

\begin{equation}
	I = 
-1\cdot 5\cdot 3\cdot \left( 5\cdot x\right) ^{2}\cdot \sin {\left( 5\cdot x\right) ^{3}}
\end{equation}
\begin{equation}
	H = 
-1\cdot 5\cdot 3\cdot \left( 5\cdot x\right) ^{2}\cdot \sin {\left( 5\cdot x\right) ^{3}}
\end{equation}
\begin{equation}
	G = 
5\cdot x^{4}\cdot \cos {x^{5}}\cdot \cos {\left( 5\cdot x\right) ^{3}}
\end{equation}
\begin{equation}
	F = 
5\cdot 3\cdot \left( 5\cdot x\right) ^{2}\cdot \cos {\left( 5\cdot x\right) ^{3}}
\end{equation}
\begin{equation}
	E = 
5\cdot 3\cdot 5\cdot 2\cdot 5\cdot x\cdot \sin {\left( 5\cdot x\right) ^{3}}
\end{equation}
\begin{equation}
	D = 
-1\cdot 5\cdot 3\cdot \left( 5\cdot x\right) ^{2}\cdot \sin {\left( 5\cdot x\right) ^{3}}
\end{equation}
\begin{equation}
	C = 
-1\cdot 5\cdot 3\cdot \left( 5\cdot x\right) ^{2}\cdot \sin {\left( 5\cdot x\right) ^{3}}
\end{equation}
\begin{equation}
	B = 
5\cdot x^{4}\cdot -1\cdot 5\cdot x^{4}\cdot \sin {x^{5}}
\end{equation}
\begin{equation}
	A = 
5\cdot 4\cdot x^{3}\cdot \cos {x^{5}}
\end{equation}
\end{document}
